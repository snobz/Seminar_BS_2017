%!TEX root = FreeRtos ARM uController.tex


\subsection{Echtzeitsysteme und Echtzeitbetriebssysteme}
\label{sec:Echtzeitsysteme}
Mit der steigenden Leis\-tungs\-fähig\-keit von modernen $\mu$Pro\-zesso\-ren steigen auch die Anforderungen an die Software, die auf diesen Systemen aufsetzt. Viele dieser Systeme fordern, trotz ihrer Komplexität, dass Teile des Pro\-gramm\-ab\-laufs in bestimmten zeitlichen Grenzen aus\-ge\-führt werden und somit vorhersehbar und deterministisch\cite{9780128015070} sind.
Systeme die solchen Anforderungen unterliegen werden Echtzeitsysteme genannt. Bezogen auf ihre Zuverlässigkeit unterliegen Echtzeitsysteme einer weiteren Unterteilung:
\begin{itemize}
	\item Echtzeitsysteme mit weicher Echtzeitanforderung (soft realtime systems)
	\item Echtzeitsysteme mit harter Echtzeitanforderung (hard realtime systems)
\end{itemize}
   Ein weiches Echtzeitsystem soll eine Aufgabe in den vorgegebenen zeitlichen Grenzen ausführen. Ein Über\-schrei\-ten der zeitlichen Grenzen ist grundsätzlich nicht erlaubt, führt aber nicht unmittelbar zu einem Fehler oder einem Versagen des Gesamtsystems. Ein hartes Echtzeitsystem muss die gestellte Aufgabe in den vorgegebenen Grenzen aus\-füh\-ren. Durch eine Über\-schrei\-tung wird das System unbrauchbar und dies führt in der Folge dazu, dass das System nicht im vorgesehenen Szenario eingesetzt werden kann. Dabei ist ausdrücklich zu beachten, dass Echtzeit nicht bedeutet, dass ein Programm besonders schnell ausgeführt wird. Die Ausführung eines Programms kann beispielsweise auch gewollt langsam sein und gerade deshalb den gestellten Echtzeitanforderung genügen. Einige Beispielsysteme und deren Echtzeitzuordnung sind in Tabelle \ref{tab:BeispieleEchtzeitsystem} aufgeführt. 
\begin{table*}
\centering
	\begin{tabular}{|l|l|l|}
		\hline
		\textbf{Beispiel} & \textbf{Echtzeit Typ}  & \textbf{Auswirkung} \\
		\hline
		Tastatur Controller & Soft Realtime & Kurzfristig verzögerte Ausgabe \\
		\hline
		Echtzeit Media Streaming  & Soft Realtime & Bild und Ton kurzfristig asynchron \\
		\hline
		Computer Numerical Control (CNC)  & Hard Realtime & Fehler bei der Fertigung des Teils\\
		\hline
		Airbag System  & Hard Realtime & Möglicher Personenschaden\\
		\hline
	\end{tabular}
	\caption{Beispiele von Echtzeitsystemen und deren Auswirkung beim Über- oder Unterschreiten der Anforderungsgrenzen}
	\label{tab:BeispieleEchtzeitsystem}
\end{table*}
Um die grund\-sätz\-liche Funktionalität eines Echtzeitbetriebssystems zu erläutern, werden zuerst die Grundmodelle für den Programmablauf eingebetteter Systeme beschrieben. Der Programmablauf lässt sich auf drei Modelle zu\-rück\-füh\-ren\cite{RTOSRevealed} (Abbildung \ref{fig:Programmablauf}). 
\begin{figure}[ht]
	\centering
		\includegraphics[width=0.3\textwidth]{Pictures/EmbeddedCom/cwrtos2f5c.jpg}
	\caption{Übersicht Programmabläufe in embedded Anwendungen. Unterscheidung von zwei Hauptkategorien: Schleifen-gesteurte Anwendungen und Event-gesteurte Anwendungen. Bild-Quelle~\protect\citeA{RTOSRevealed}}
	\label{fig:Programmablauf}
\end{figure}
Eingebettete Anwendungen können in einer endlosen Hauptschleife laufen (mit und ohne Interruptunterbrechungen), in der die Anweisungen sequenziell abgearbeitet werden, oder aber in event-gesteuerten ne\-ben\-läuf\-igen ei\-gen\-stän\-dig\-en Pro\-gramm\-ab\-schnit\-ten (Thre\-ad oder Task\footnote{Nachfolgenden wird Task benutzt, da dies der geläufige Begriff bei FreeRTOS ist. In der Literatur zu Echtzeitsystemen ist der Begriff nicht exakt definiert.}). Die ne\-ben\-läuf\-ige Aus\-füh\-rung der unterschiedlichen Programmsegmente ist nur durch einen Scheduler, welcher Teil eines RTOS Kernels ist, zu erreichen. Der RTOS Kernel abstrahiert Timinginformationen\cite{MasteringFreeRtos} 
%Chrstoph: Ein RTOS Kernel ermöglicht eine Abstraktion von der zugrunde liegenden Hardware, indem ein Teil der hardwarespezifischen Steuerkommandos durch den Kernel ausgeführt werden. Tasks können auf die Funktionen des Kernels zur Steuerung dieser Hardwarefunktionen zu\-rück\-grei\-fen. Außerdem ermöglicht der Kernel eine weitergehende Steuerung, beispielsweise durch die Verwaltung von Timing Informationen.
%Michael: Ich habe das entfernt, wie auch schon von Frau Ma angesprochen trifft das bei einem RTOS nicht zu. Die Abstraktion zur Hardware wird durch die HAL (Hardware Abstraction Layer) ermöglicht. FreeRTOS stell von sich aus keine Hardware Funtkionalitäten zur Verfügung. FreeRTOS nutzt eigentlich keine hardware spezifischen Anweisungen, bis auf den Tick Interrupt. In einigen Portierungen werden spezielle prozessorspezifische assambler Befehle genutzt um den Task switch besonders effizient zu gestalten. Aus einer Task oder in einer Task werden Hardware spezifischen Kommandos verwendet. 
und stellt durch den Scheduler sicher, dass der näch\-ste Task rechtzeitig ausgeführt wird. Der Entwickler ist dafür verantwortlich, dass der Task die ge\-wün\-schte Aufgabe im zeitlichen Rahmen ausführt. 
Für viele kleine Anwendungen kann die Ausführung in einer endlosen Hauptschleife durchaus sinnvoll sein, wenn beispielsweise die Ressourcen so knapp sind, dass ein Overhead durch zusätzliche Verwaltungsfunktionen ausgeschlossen werden muss. Ein großer Nachteil der "`endlosen einschleifigen Variante"' ist die permanente Nutzung des Prozessors, auch "`processor hogging"' oder "`CPU hogging"' genannt. Um den Prozessor in dieser Variante in einen Energiesparmodus zu versetzen, muss durch den Entwickler sichergestellt werden, dass das Gesamtsystem alle Anweisungen ausgeführt hat und bereit ist schlafen zu gehen. Bei komplexen Anwendungen mit vielen Ab\-häng\-ig\-kei\-ten kann es so zu erheblichen Implementierungsaufwand kommen.
% Christoph: Um den Prozessor in dieser Variante in einen Energiesparmodus zu versetzen, sind umfangreiche Kenntnisse über den Prozessor erforderlich.
% Michael: Das ist so nicht richtig. Der Ablauf wie man den Prozessor in den Schlafmodus bringt, ist bei beiden identisch. Dies wird wie oben beschrieben durch die HAL erldigt. Das einzige was anders ist, ist dass man nicht so einfach feststellen kann, wann man schlafen gehen kann.
%Christoph: Außerdem müssen sich Entwickler an eine sehr strukturierte Programmierung halten, die gerade bei zukünftigen Anpassungen der Software zu Problemen führen kann.
%Michael:Man muss doch immer strukturiert Programmieren :) . Ich glaube nicht das die strukturierte Programmierung hier das Problem ist, sondern die Abhängigkeiten, wann das Systemschlafen gehen kann.Das mit den Anpassungen steht bereits weiter untem im Text
%Christoph: Der ursprüngliche Satz passte hier nicht so richtig, schau mal, ob das so besser passt
%Michael: Das passt
Besonders bei akkubetriebenen Geräten wie IoT Devices oder Mobiltelefonen wird sehr genau auf die Energieaufnahme geachtet, sodass die Nutzung einer einzigen Hauptschleife hier nicht effektiv genug ist. Ein RTOS Kernel arbeitet mit einem Event gesteuerten Programmablauf, ein "`CPU hogging"' kann somit vermieden werden. Des Weiteren bieten viele RTOS Kernel sehr einfache Lö\-sun\-gen zur effektiven Nutzung von Energiesparmodi. Dies wird in Abschnitt \ref{sec:Low Power Modes} am Beispiel von FreeRTOS und einem ARM $\mu$Prozessor demonstriert. Neben der Echt\-zeit\-fähig\-keit gibt es aber noch viele weitere Vorzüge für den Einsatz eines Echtzeitbetriebssystems. Durch das Herunterbrechen der Anwendungen in Tasks entstehen viele kleine Module, die jeweils eine kleine Teilaufgabe des Gesamtsystems über\-neh\-men. Durch ein sauber definiertes Interface zur Kommunikation zwischen den Tasks, lässt sich die Entwicklungsarbeit gut auf mehrere Teams verteilen. Dies ermöglicht auch den Einsatz von agilen Entwicklungsmethoden wie Scrum in der Entwicklung von eingebetteten Systemen. Ein weiterer großer Vorteil ist die Erweiterbarkeit von RTOS Anwendungen. Bei Änderungen von Anwendungen, die in einer Schleife laufen, ist oft der gesamte Code von dieser Änderungen betroffen. Ein RTOS hat durch die Interprozesskommunikation eine natürliche Lose-Kopplung zwischen den einzelnen Programmfunktionalitäten. Das Än\-dern oder Hinzufügen von Tasks ist somit wesentlich einfacher, da andere Tasks nicht unmittelbar durch diese Än\-der\-ung betroffen sind. 