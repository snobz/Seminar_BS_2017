%!TEX root = FreeRtos ARM uController.tex

%\subsection{Anforderung Desktop Betriebssystem vs. Anforderung Echtzeit Betriebssystem}
%Desktop Betriebssysteme verwalten den Hardwarezugriff und stellen sicher, dass eingesetzte Software die benötigte Rechenzeit zur Verfügung gestellt bekommt. Gleichzeitig regeln Sie den Hardwarezugriff und organisieren den konkurrierenden Zugriff, beispielsweise auf Netzwerkkarten und Festplatten. Sie stellen Funktionen für die Interprozesskommunikation bereit und übernehmen grundlegende Aufgaben wie die Organisation von Arbeitsspeicher. Im Gegensatz zu einem gewöhnlichen Desktop Betriebssystem liegt der Anwendungsfokus bei Echzeit Betriebssystemen nicht auf der direkten Userinteraktion. Die wenigsten Echtzeitbetriebssysteme bieten eine Benutzeroberfläche. Die Interaktion mit der Umwelt geschieht durch spezielle Hardware und ist zumeist für genau eine Aufgabe ausgelegt. Viele Funktionen die eine Desktop Betriebssystem übernimmt wie z.B. das Verwalten von konkurrierenden Zugriffen auf externe Geräte müssen vom Entwickler selbst übernommen werden. Ein Echtzeitbetriebssystem bietet bei weitem nicht die Funktionalitäten eines Desktop Betriebssystems. Das liegt zum Einen an der Art von Zielsystemen, die zumeist nur einen beschränkten Speicher aufweisen und Zweitens an der Tatsache, dass die vom Echtzeitbetriebssystem bereitgestellten Funktionen deterministisch sein müssen.   
%
%WIP: Gefällt insgesamt noch nicht, benötigen wir Anforderungen an ein Destkop Betriebssystem ?? Echzeitbetriebsysteme unterscheiden sich erheblich von normalen Betriebssystemen. Wie schon angesprochen verwaltet ein RTOS nur die Zeit (Scheduler) + Speicher (Memory Allocation) 
%
\subsection{Echtzeitsysteme und Echtzeitbetriebssysteme}
\label{sec:Echtzeitsysteme}
Mit der steigenden Leis\-tungs\-fähig\-keit von modernen $\mu$\-Pro\-zesso\-ren, steigen auch die Anforderungen an die Software die auf diese Systeme aufsetzt. Viele dieser Systeme verlangen trotz ihrer Komplexität, dass Teile des Programmablauf in bestimmten zeitlichen Grenzen ausgeführt wird und somit vorhersehbar, deterministisch und zuverlässig sind. Systeme die solchen Anforderungen unterliegen werden Echtzeitsysteme genannt. Echtzeitsysteme unterliegen einer weiteren Unterteilung in Echtzeitsysteme mit weicher Echtzeitanforderungen (soft realtime systems), nachfolgend weiches Echtzeitsystem genannt, und Echtzeitsysteme harter Echtzeitanforderung (hard realtime systems), nachfolgend hartes Echtzeitsystem genannt. Ein weiches Echtzeitsystem soll eine Aufgabe in den vorgegeben zeitlichen Grenzen ausführen, ein über\-schreiten der zeitlichen Grenzen ist grundsätzlich nicht erlaubt und führt nicht unmittelbar zu einem Fehler oder einem Versagen des Gesamtsystems. Ein hartes Echtzeitsystem hingegen muss die gestellte Aufgabe in den vorgegebenen Grenzen aus\-füh\-ren. Durch eine Überschreitung wird das System unbrauchbar und führt dazu, dass das System nicht im vorgesehenen Szenario eingesetzt werden kann. Einige Beispielsysteme und deren Echtzeitzuordnung wird in Tabelle \ref{tab:BeispieleEchzeitsystem} gezeigt. Um die grundsätzliche Funktionalität eines Echtzeitbetriebssystems zu erläutern, werden zuerst die Grundmodelle für den Programmablauf eingebetteter Systeme beschrieben werden. Der Programmablauf eingebetteter Systeme lässt sich auf drei Modelle zurückführen (Abbildung \ref{fig:Programmablauf}). Eingebettete Anwendungen können in einer einzigen Schleife (mit oder ohne Interrupt Unterbrechungen) laufen oder aber in event-gesteuerten ne\-ben\-läuf\-igen ei\-gen\-stän\-dig\-en Programmabschnitten (Thre\-ad oder Task\footnote{Nachfolgenden wird Task benutzt, da dies der geläufige Begriff bei FreeRTOS ist. In der Literatur zu Echtzeitsystemen ist der Begriff nicht exakt definiert.}) ausgeführt werden. Die nebenläufige Aus\-füh\-rung der unterschiedlichen Programmsegmente ist nur durch einen geeigneten Scheduler, welcher Teil eines RTOS Kernels ist, zu erreichen. Ein RTOS Kernel abstrahiert von der zugrunde liegenden Hardware und ermöglicht weitergehende Steuerung, beispielsweise durch Verwaltung von Timing Informationen. Hierdurch kann sichergestellt werden, dass die nächste Task rechtzeitig ausgeführt wird. Der Entwickler ist dafür verantwortlich, dass die Task die gewünschte Aufgabe im zeitlichen Rahmen ausführt. Durch den Einsatz des RTOS Kernels kann er jedoch auf Spezifika der Hardware verzichten und die Funktionen des Kernels verwenden. Wie sichergestellt werden kann, dass eine Task  harten oder weichen Echtzeitanforderungen entspricht wird Abschnitt \ref{sec:Echtzeitanalyse} beschrieben. Für viele kleine Anwendungen kann die Nutzung einer einzigen Schleife durchaus sinnvoll sein, wenn beispielsweise die Ressourcen so knapp sind, dass ein Overhead durch zusätzliche Verwaltungsfunktionen ausgeschlossen werden muss. Ein großer Nachteil der "`einschleifen Variante"' ist die permanente Nutzung des Prozessors. Um den Prozessor in dieser Variante in einen Energiesparmodus zu versetzen sind umfangreiche Kenntnisse über den Prozessor sowie eine sehr strukturierte Programmierung erforderlich, die gerade bei Anpassungen der Software zu Problemen führen kann. Besonders bei akkubetriebenen Geräten wie IoT Devices oder Mobiltelefonen wird sehr genau auf die Energieaufnahme geachtet. Ein RTOS bietet Funktionen zur Prüfung an, mit denen beispielsweise ermittelt werden kann, ob die Versetzung in den Schlafmodus möglich ist. Dies wird in Abschnitt \ref{sec:Low Power Modes} am Beispiel von FreeRTOS und einem ARM $\mu$Prozessor demonstriert. Neben der Echtzeitfähig gibt es aber noch viele weitere Vorzüge für den Einsatz eines Echtzeitbetriebssystems.  
Durch das Herunterbrechen der Anwendungen in Tasks entstehen viele kleine Module, die jeweils eine kleine Teilaufgabe des Gesamtsystems übernehmen. Durch ein sauber definiertes Interface zur Kommunikation der Tasks lässt sich die Entwicklungsarbeit leicht auf mehrere Teams verteilen. Dies ermöglicht auch den Einsatz von agilen Entwicklungsmethoden wie Scrum in der Entwicklung von eingebetteten Systemen. 
To be continued ... ;)  
\newline  
\begin{figure}
	\centering
		\includegraphics[width=0.3\textwidth]{Pictures/EmbeddedCom/cwrtos2f5c.jpg}
	\caption{Übersicht Programmabläufe in embedded Anwendungen. Quelle~\protect\citeA{RTOSRevealed}}
	\label{fig:Programmablauf}
\end{figure}

\begin{table*}
\centering
	\begin{tabular}{|l|l|l|}
		\hline
		\textbf{Beispiel} & \textbf{Echtzeit Typ}  & \textbf{Auswirkung} \\
		\hline
		Tastatur Controller & Soft Realtime & kurzfristig verzögerte Ausgabe \\
		\hline
		Echtzeit Media Streaming  & Soft Realtime & Bild und Ton kurzfristig asynchron \\
		\hline
		Controller CD Laufwerk  & Hard Realtime & Fehler beim Lesen\\ % Wirklich harte Forderung? Ich könnte das lesen ja wiederholen. Beim Brennen wäre dies ein größeres Problem, da die Spur dann unterbrochen ist.
		\hline
		Airbag System  & Hard Realtime & möglicher Personenschaden\\
		\hline
	\end{tabular}
	\caption{Beispiele Echzeitsystem}
	\label{tab:BeispieleEchzeitsystem}
\end{table*}




%Echtzeitbetriebssysteme kommen zum Einsatz wenn neben den oben genannten Anforderungen an ein normales Betriebssystem weitere Anforderungen gestellt werden, die ein normales Betriebssystem nicht berücksichtigt. Dies können beispielsweise garantiert berechenbare Reaktionszeiten sein wie sie in der Fabrikation oder im Automobilbereich gefordert werden oder geringe Leistungsaufnahmen wie bei Komponenten des Internet of Things (IoT). Insgesamt wird zwischen Harten und Weichen Echtzeitkriterien unterschieden. Diese Gliedern sich wie folgt:\newline
%Aufgrund der Eingangs geschilderten Einsatzbereiche ist leich zu erkennen, dass Echzeitbetriebssysteme häufigin Umgebungen zum Einsatz kommen, in denen besondere Anforderungen an die Hardware gestellt werden. Häufig verfügt die Hardware nur über begrenzte Speicherkapazitäten, über geringe Wärmeableitfähigkeiten und damit geringe Rechenleistung. Die zur Verfügung stehende Energie muss bei der Entwicklung ebenfalls berücksichtigt werden. 
%Vor diesem Hintergrund benötigen Echtzeitbetriebssysteme nur wenig Speicherplatz und implementieren Funktionen um den Prozessor und die angeschlossene Peripherie nur kurzzeitig zu belasten und in der restlichen Zeit in den Ruhezustand zu verseten.
