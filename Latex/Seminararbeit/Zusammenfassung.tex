%!TEX root = FreeRtos ARM uController.tex
%\pagebreak
\section{Zusammenfassung}
FreeRTOS kann als freies professionelles Echtzeitbetriebssystem betrachtet werden. Es steht den kommerziellen Echtzeitbetriebssystemen in Sachen Funktionalität in nichts nach. Bei den Herstellern von uControllern und ISP ist FreeRTOS eines des Standard Echtzeitbetriebssysteme. Es stehen gewöhnlich viele Beispiele oder Template Projekte für FreeRTOS zur Verfügung.  Besonders für Einsteiger ist FreeRTOS sehr zu empfehlen, da es kostenlos und sehr gut dokumentiert ist. Da FreeRTOS offenen Source Code zur Verfügung stellt, ist es dem Entwickler auch möglich einen Blick in die Implementierung des Echtzeitsystems zu werfen. Was besonders beim Verstehen des Kernels hilfreich ist. Da FreeRTOS auch in einer kommerziellen Version angeboten wird, kann davon ausgegangen werden, dass der Kernel auch langfristig Support erfährt. Ein Nachteil ist die komplizierte Einrichtung einer freien Entwicklungsumgebung wie Eclipse CDT. Es bedarf enormen Konfigurationsaufwand bis eine Beispielanwendung mit FreeRTOS auf dem Zielsystem läuft.     