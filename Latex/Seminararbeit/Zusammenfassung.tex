%!TEX root = FreeRtos ARM uController.tex
%\pagebreak
\section{Zusammenfassung}
FreeRTOS kann als freies, professionelles Echtzeitbetriebssystem betrachtet werden. Im Vergleich zu kommerziellen Echtzeitbetriebssystemen wird ein annähernd gleicher Funktionsumfang gewährleistet. Bei den Herstellern von uControllern und ISP ist FreeRTOS eines der Standard Echtzeitbetriebssysteme. Es stehen gewöhnlich viele Beispiele oder Template Projekte für FreeRTOS zur Verfügung.  Besonders für Einsteiger ist FreeRTOS sehr zu empfehlen, da es kostenlos und sehr gut dokumentiert ist. FreeRTOS wird als offener Source Code zur Verfügung gestellt. Hierdurch ist es dem Entwickler auch möglich einen Blick in die Implementierung des Echtzeitsystems zu werfen. Dies ist besonders beim Verstehen des Kernels hilfreich. Da FreeRTOS auch in einer kommerziellen Version angeboten wird, kann davon ausgegangen werden, dass der Kernel auch langfristig Support erfährt. Ein Nachteil ist die komplizierte Einrichtung einer freien Entwicklungsumgebung wie Eclipse CDT. Um eine erste Beispielanwendung mit FreeRTOS auf dem Zielsystem zu implementieren, müssen diverse Konfigurationen an der Entwicklungsumgebung vorgenommen werden.
