% !TEX encoding = UTF-8 Unicode

% Beispiel für ein LaTeX-Dokument im Format "seminarvorlage"
\documentclass[ngerman]{seminarvorlage}
% ngerman = Deutsch in neuer Rechtschreibung, alternativ english
\usepackage[utf8]{inputenc} % Kodierung der Umlaute
\usepackage{babel} % automatische Sprachanpassung, Sprache siehe oben
\usepackage{cleveref} % für bequeme Referenzen, siehe \cref unten

%Own Code
	
\usepackage{amsmath}
\usepackage{listings}
\usepackage{xcolor}
\lstset { %
    language=C++,
		commentstyle=\color{gray},
    backgroundcolor= \color{black!4}, % set backgroundcolor
    basicstyle= \small,%\tiny,%\footnotesize,% basic font setting
		showtabs=false,
		tabsize=1,
		breaklines=true,
		numbers=left,
		 showspaces=false,
		showtabs=false,
		keepspaces=true,
				 numbersep=1pt,
  showstringspaces=false,
	aboveskip=5pt,
belowskip= -6pt
}

\newcommand*{\fullref}[1]{\hyperref[{#1}]{\autoref*{#1} \nameref*{#1}}} % One single link
\newcommand*{\quelle}{% 
  \footnotesize Quelle: 
} 

\begin{document}
% Unbedingt angeben: Titel, Autoren, E-Mail
% Freiwillig: Adresse
\title{Embedded Realtime OS FreeRTOS auf STM32F4}
\numberofauthors{2}
\author{
  \alignauthor Michael Ebert\\
		\affaddr{Ad-hoc Networks GmbH}\\
    \email{ebert@ad-hoc.network}
  \alignauthor Christoph Bläßer\\
		\affaddr{Bundesamt für Sicherheit in der Informationstechnik}
    \email{christoph.blaesser@gmx.de}
}

\maketitle
\keywords{ FreeRTOS, RTOS, ARM , STM32, Real Time.}
\abstract{
Im Rahmen dieser Arbeit wird das Echtzeitbetriebssystem FreeRTOS vorgestellt. Hierzu werden zu Beginn die allgemeinen Eigenschaften für Echtzeitbetriebssysteme beschrieben. Im Verlauf des Textes wird an ausgewählten Beispielen dargestellt, wie FreeRTOS diese Anforderungen berücksichtigt und durch geeignete Programmfunktionen umsetzt.
}
\section{Grundlagen Echtzeitsysteme}
%!TEX root = FreeRtos ARM uController.tex


\subsection{Echtzeitsysteme und Echtzeitbetriebssysteme}
\label{sec:Echtzeitsysteme}
Mit der steigenden Leis\-tungs\-fähig\-keit von modernen $\mu$Pro\-zesso\-ren steigen auch die Anforderungen an die Software, die auf diesen Systemen aufsetzt. Viele dieser Systeme fordern, trotz ihrer Komplexität, dass Teile des Pro\-gramm\-ab\-laufs in bestimmten zeitlichen Grenzen aus\-ge\-führt werden und somit vorhersehbar und deterministisch\cite{9780128015070} sind.
Systeme die solchen Anforderungen unterliegen werden Echtzeitsysteme genannt. Bezogen auf ihre Zuverlässigkeit unterliegen Echtzeitsysteme einer weiteren Unterteilung:
\begin{itemize}
	\item Echtzeitsysteme mit weicher Echtzeitanforderung (soft realtime systems)
	\item Echtzeitsysteme mit harter Echtzeitanforderung (hard realtime systems)
\end{itemize}
   Ein weiches Echtzeitsystem soll eine Aufgabe in den vorgegebenen zeitlichen Grenzen ausführen. Ein Über\-schrei\-ten der zeitlichen Grenzen ist grundsätzlich nicht erlaubt, führt aber nicht unmittelbar zu einem Fehler oder einem Versagen des Gesamtsystems. Ein hartes Echtzeitsystem muss die gestellte Aufgabe in den vorgegebenen Grenzen aus\-füh\-ren. Durch eine Über\-schrei\-tung wird das System unbrauchbar und dies führt in der Folge dazu, dass das System nicht im vorgesehenen Szenario eingesetzt werden kann. Dabei ist ausdrücklich zu beachten, dass Echtzeit nicht bedeutet, dass ein Programm besonders schnell ausgeführt wird. Die Ausführung eines Programms kann beispielsweise auch gewollt langsam sein und gerade deshalb den gestellten Echtzeitanforderung genügen. Einige Beispielsysteme und deren Echtzeitzuordnung sind in Tabelle \ref{tab:BeispieleEchtzeitsystem} aufgeführt. 
\begin{table*}
\centering
	\begin{tabular}{|l|l|l|}
		\hline
		\textbf{Beispiel} & \textbf{Echtzeit Typ}  & \textbf{Auswirkung} \\
		\hline
		Tastatur Controller & Soft Realtime & Kurzfristig verzögerte Ausgabe \\
		\hline
		Echtzeit Media Streaming  & Soft Realtime & Bild und Ton kurzfristig asynchron \\
		\hline
		Computer Numerical Control (CNC)  & Hard Realtime & Fehler bei der Fertigung des Teils\\
		\hline
		Airbag System  & Hard Realtime & Möglicher Personenschaden\\
		\hline
	\end{tabular}
	\caption{Beispiele von Echtzeitsystemen und deren Auswirkung beim Über- oder Unterschreiten der Anforderungsgrenzen}
	\label{tab:BeispieleEchtzeitsystem}
\end{table*}
Um die grund\-sätz\-liche Funktionalität eines Echtzeitbetriebssystems zu erläutern, werden zuerst die Grundmodelle für den Programmablauf eingebetteter Systeme beschrieben. Der Programmablauf lässt sich auf drei Modelle zu\-rück\-füh\-ren\cite{RTOSRevealed} (Abbildung \ref{fig:Programmablauf}). 
\begin{figure}[ht]
	\centering
		\includegraphics[width=0.3\textwidth]{Pictures/EmbeddedCom/cwrtos2f5c.jpg}
	\caption{Übersicht Programmabläufe in embedded Anwendungen. Unterscheidung von zwei Hauptkategorien: Schleifen-gesteurte Anwendungen und Event-gesteurte Anwendungen. Bild-Quelle~\protect\citeA{RTOSRevealed}}
	\label{fig:Programmablauf}
\end{figure}
Eingebettete Anwendungen können in einer endlosen Hauptschleife laufen (mit und ohne Interruptunterbrechungen), in der die Anweisungen sequenziell abgearbeitet werden, oder aber in event-gesteuerten ne\-ben\-läuf\-igen ei\-gen\-stän\-dig\-en Pro\-gramm\-ab\-schnit\-ten (Thre\-ad oder Task\footnote{Nachfolgenden wird Task benutzt, da dies der geläufige Begriff bei FreeRTOS ist. In der Literatur zu Echtzeitsystemen ist der Begriff nicht exakt definiert.}). Die ne\-ben\-läuf\-ige Aus\-füh\-rung der unterschiedlichen Programmsegmente ist nur durch einen Scheduler, welcher Teil eines RTOS Kernels ist, zu erreichen. Der RTOS Kernel abstrahiert Timinginformationen\cite{MasteringFreeRtos} 
%Chrstoph: Ein RTOS Kernel ermöglicht eine Abstraktion von der zugrunde liegenden Hardware, indem ein Teil der hardwarespezifischen Steuerkommandos durch den Kernel ausgeführt werden. Tasks können auf die Funktionen des Kernels zur Steuerung dieser Hardwarefunktionen zu\-rück\-grei\-fen. Außerdem ermöglicht der Kernel eine weitergehende Steuerung, beispielsweise durch die Verwaltung von Timing Informationen.
%Michael: Ich habe das entfernt, wie auch schon von Frau Ma angesprochen trifft das bei einem RTOS nicht zu. Die Abstraktion zur Hardware wird durch die HAL (Hardware Abstraction Layer) ermöglicht. FreeRTOS stell von sich aus keine Hardware Funtkionalitäten zur Verfügung. FreeRTOS nutzt eigentlich keine hardware spezifischen Anweisungen, bis auf den Tick Interrupt. In einigen Portierungen werden spezielle prozessorspezifische assambler Befehle genutzt um den Task switch besonders effizient zu gestalten. Aus einer Task oder in einer Task werden Hardware spezifischen Kommandos verwendet. 
und stellt durch den Scheduler sicher, dass der näch\-ste Task rechtzeitig ausgeführt wird. Der Entwickler ist dafür verantwortlich, dass der Task die ge\-wün\-schte Aufgabe im zeitlichen Rahmen ausführt. 
Für viele kleine Anwendungen kann die Ausführung in einer endlosen Hauptschleife durchaus sinnvoll sein, wenn beispielsweise die Ressourcen so knapp sind, dass ein Overhead durch zusätzliche Verwaltungsfunktionen ausgeschlossen werden muss. Ein großer Nachteil der "`endlosen einschleifigen Variante"' ist die permanente Nutzung des Prozessors, auch "`processor hogging"' oder "`CPU hogging"' genannt. Um den Prozessor in dieser Variante in einen Energiesparmodus zu versetzen, muss durch den Entwickler sichergestellt werden, dass das Gesamtsystem alle Anweisungen ausgeführt hat und bereit ist schlafen zu gehen. Bei komplexen Anwendungen mit vielen Ab\-häng\-ig\-kei\-ten kann es so zu erheblichen Implementierungsaufwand kommen.
% Christoph: Um den Prozessor in dieser Variante in einen Energiesparmodus zu versetzen, sind umfangreiche Kenntnisse über den Prozessor erforderlich.
% Michael: Das ist so nicht richtig. Der Ablauf wie man den Prozessor in den Schlafmodus bringt, ist bei beiden identisch. Dies wird wie oben beschrieben durch die HAL erldigt. Das einzige was anders ist, ist dass man nicht so einfach feststellen kann, wann man schlafen gehen kann.
%Christoph: Außerdem müssen sich Entwickler an eine sehr strukturierte Programmierung halten, die gerade bei zukünftigen Anpassungen der Software zu Problemen führen kann.
%Michael:Man muss doch immer strukturiert Programmieren :) . Ich glaube nicht das die strukturierte Programmierung hier das Problem ist, sondern die Abhängigkeiten, wann das Systemschlafen gehen kann.Das mit den Anpassungen steht bereits weiter untem im Text
%Christoph: Der ursprüngliche Satz passte hier nicht so richtig, schau mal, ob das so besser passt
%Michael: Das passt
Besonders bei akkubetriebenen Geräten wie IoT Devices oder Mobiltelefonen wird sehr genau auf die Energieaufnahme geachtet, sodass die Nutzung einer einzigen Hauptschleife hier nicht effektiv genug ist. Ein RTOS Kernel arbeitet mit einem Event gesteuerten Programmablauf, ein "`CPU hogging"' kann somit vermieden werden. Des Weiteren bieten viele RTOS Kernel sehr einfache Lö\-sun\-gen zur effektiven Nutzung von Energiesparmodi. Dies wird in Abschnitt \ref{sec:Low Power Modes} am Beispiel von FreeRTOS und einem ARM $\mu$Prozessor demonstriert. Neben der Echt\-zeit\-fähig\-keit gibt es aber noch viele weitere Vorzüge für den Einsatz eines Echtzeitbetriebssystems. Durch das Herunterbrechen der Anwendungen in Tasks entstehen viele kleine Module, die jeweils eine kleine Teilaufgabe des Gesamtsystems über\-neh\-men. Durch ein sauber definiertes Interface zur Kommunikation zwischen den Tasks, lässt sich die Entwicklungsarbeit gut auf mehrere Teams verteilen. Dies ermöglicht auch den Einsatz von agilen Entwicklungsmethoden wie Scrum in der Entwicklung von eingebetteten Systemen. Ein weiterer großer Vorteil ist die Erweiterbarkeit von RTOS Anwendungen. Bei Änderungen von Anwendungen, die in einer Schleife laufen, ist oft der gesamte Code von dieser Änderungen betroffen. Ein RTOS hat durch die Interprozesskommunikation eine natürliche Lose-Kopplung zwischen den einzelnen Programmfunktionalitäten. Das Än\-dern oder Hinzufügen von Tasks ist somit wesentlich einfacher, da andere Tasks nicht unmittelbar durch diese Än\-der\-ung betroffen sind.  
\section{FreeRTOS} 
%!TEX root = FreeRtos ARM uController.tex
\subsection{Geschichte}
FreeRTOS wird seit etwa 10 Jahren von der Firma Real Time Engineers Ltd. in Zusammenarbeit mit verschiedenen Chipherstellern entwickelt. Derzeit unterstützt es 35 Architekturen und wurde mehr als 113000 mal heruntergeladen. Das Entwickler Team unter Führung des Gründers Richard Barry, konzentrieren sich bei der Entwicklung darauf sowohl ein geeignetes Qualitätsmanagement umzusetzen, als auch die Verfügbarkeit der verschiedenen Dateiversionen zu gewährleisten. FreeRTOS wird in zwei verschiedenen Lizenzmodellen angeboten, die eine Anpassung der originären GNU General Public Licence darstellen. Die Open Source Lizenz (FreeRTOS) erhält keine Garantien und keinen direkten Support. Entwickler, die diese freie Lizenz verwenden und Än\-der\-ungen am RTOS Kernel vornehmen müssen den Quellcode ihrer Än\-der\-ungen für die Community offenlegen. In der kommerziellen Lizenz (SafeRTOS) kann selbst entwickelter Code als closed source vertrieben werden. Ebenso unterstützt Real Time Engineers Ltd. bei der Entwicklung und bietet entsprechende Garantie für die Echtzeitfähigkeit von FreeRTOS. Real Time Engineers bietet zu FreeRTOS diverse Erweiterungen wie Treiber und Tools. Geführt werden diese Erweiterungen unter dem Namen FreeRTOS Ecosystem, dazu gehören unter anderem ein FAT Dateisystem, TCP/ UDP Stacks, sowie TLS/SSL Implementierungen. 
%!TEX root = FreeRtos ARM uController.tex
\subsection{Entwicklungsumgebung}
FreeRTOS ist im Prinzip nicht an eine spezielle Entwicklungsumgebung gebunden. Bevor eine Entwicklung be\-ginnt ist es dennoch ratsam sich einen Überblick über die ver\-fügbaren IDEs\footnote{Integrated Development Environment} zu machen. Der wichtigste Punkt der hier zu nennen ist, ist das Debugging. Da ein Echtzeitbetriebssystem eine weitere Abstraktionsebene hinzufügt und wie eine Art Middleware fungiert, lassen Sich viele RTOS spezifische Funktionen und Eigenschaften wie Queues, Task Stacks etc. nur mühsam mit einem Debugger wie GDB untersuchen. Viele der marktgängigen Entwicklungsumgebungen bieten daher spezielle RTOS-aware Pakete, so dass ein einfacherer Zugriff auf RTOS Objekte und Eigenschaften möglich ist. Wie die RTOS awareness beim Debugging eingesetzt wird und welche Funktionalitäten sie einem Entwickler bietet wird in Abschnitt \ref{sec:Debugging von Echtzeitsystemen} aufgezeigt. Ein weiterer Punkt der bei der Auswahl der IDE getroffen werden muss sind die Kosten. Bei Propritäre IDEs können oft mehrere tausend Euro Lizenzkosten anfallen, bieten aber den Vorteil der nahtlosen Einbindungen von $\mu$Prozessoren und Echtzeitbetriebssystem (RTOS awareness). Bei der Entwicklung von ARM uProzessoren sind hier Keil (Arm), IAR Workbench und True Studio (Atollic) zu nennen. Diese Entwicklungsumgebungen lassen sich zum Teil auch frei verwenden, allerdings mit starken Einschränkungen wie z.B. maximal Codesize. Auf der nicht proprietären Seiten steht Eclipse CDT zur Verfügung, es ist komplett frei in der Verwendung und hat keine Beschränkungen. Nachteil ist hier, dass die Integration nicht so einfach ist wie bei den proprietären IDEs. RTOS awarness wird bei Eclipse durch die Installation weiterer Plugins erreicht. Ein weiterer Nachteil ist, dass es keine Beispielprojekte für Eclipse CDT und FreeRTOS zur Ver\-fü\-gung stehen, daher müssen Projekte von Grund auf selbst konfiguriert und installiert werden. Da im Laufe dieser Arbeit Eclipse CDT für alle Beispiele verwendet wird, wird in Abschnitt \ref{sec:Einrichtung und Konfiguration} das Aufsetzen einer Basiskonfiguration erklärt. 

%!TEX root = FreeRtos ARM uController.tex
\subsection{Zielsysteme STM32F4 (ARM Cortex M3)}
32 bit Prozessor - Funktionsübersicht, Hinweis Port Teil von FreeRTOS
%!TEX root = FreeRtos ARM uController.tex
\pagebreak
\subsection{Einrichten und Konfiguration}
\label{sec:Einrichtung und Konfiguration}
Ausgangspunkt für die nachfolgenden Codebeispiele ist die derzeit aktuelle Entwicklungsumgebung Eclipse Neon. Diese wird in der C/C++ Variante (CDT) auf dem Entwicklungssystem (Windows 7 Professional) installiert. Im Anschluss muss das GNU ARM Plugin für Eclipse CDT installiert werden, dies ist entweder über den Pluginmanager oder über den folgenden Link erhältlich: 
\newline
\newline
http://gnuarmeclipse.github.io/
\newline
\newline
Das Plugin ermöglicht die Einbindung und die Konfiguration von ARM Cross Compilern. Des Weiteren stellt es einige Beispielprojekte für ARM uController zur Verfügung. Nach der Installation des ARM Plugins, müssen die GCC ARM Toolchain und die GNU Build Tools installiert werden. 
Die Toolchain kann hier heruntergeladen werden: 
\newline
\newline
https://launchpad.net/gcc-arm-embedded
\newline
\newline
Die Toolchain und die Buildtools stellen nötigen Anwendungen die zum Compilieren und Debuggen der C und C++ Files benötigt werden. Zur Toolchain gehören unter anderem GCC als Cross Compiler und GDB (GNU Debugger) zum Debuggen der Anwendung auf der Zielplattform. GNU Buildtools beinhalte make und rm, die zum Organisieren des Builds benötigt werden. Nach der Installation müssen die Verzeichnisse der Toolchain und der Buildtools im Plugin konfiguriert werden. Mit dieser Konfiguration ist das System nun in der Lage C und C++ Dateien für die Zielplattform zu kompilieren und als Binary File (.elf) bereitzustellen. Zum Übertragen und Debuggen der Anwendung auf dem Zielsystem wird ein ISP-Programmer für ARM benötigt. Folgende ISP-Programmer werden häufig verwendet. Diese Liste ist nicht vollständig und stellt auch keine Empfehlung dar. 
\begin{itemize}
	\item Segger J-Link:
	\newline
	https://www.segger.com/jlink-debug-probes.html
	\item Keil Ulink: 
	\newline
	http://www2.keil.com/mdk5/ulink
	\item STM ST-Link/VL: 
	\newline
	http://www.st.com/en/development-tools/st-link-v2.html
\end{itemize}
Zur Nutzung des ISP müssen die benötigten Treiber und On-Chip Debugger des Herstellers installiert werden.            
Nachdem die Basiskonfiguration abgeschlossen ist, kann nun eine Basis Projekt erstellt werden. Hierfür verwendet man am Besten ein Templateprojekt des GNU ARM Plugins, siehe Abbildung \ref{fig:NewProj}.
\begin{figure}[htb]
	\centering
		\includegraphics[width=0.4\textwidth]{Pictures/Einrichtung/NewF4Project.png}
	\caption{Erstellung eines Basisprojekts für den STM32F4 durch das GNU ARM Plugin.}
	\label{fig:NewProj}
\end{figure}
Das Templateprojekt beinhaltet bereits alle benötigten Hardware Librarys wie die STM HAL (siehe Abschnitt \ref{sec:Zielsysteme}) oder das ARM CMSIS (Cortex Microcontroller Software Interface Standard).
\begin{figure}[htb]
	\centering
		\includegraphics[width=0.4\textwidth]{Pictures/Einrichtung/CMSISv4_small.jpg}
	\caption{CMSIS}
	\label{fig:CMSIS}
\end{figure}
Das Templateprojekt sollte jetzt kompilieren und mittels ISP-Programmer auf auf dem Zielsystem ausgeführt werden können.
Als nächstes wird FreeRTOS in das Templateprojekt eingebunden, hierfür kann auf www.freertos.org die gepackte Variante der Demoprojekte heruntergeladen werden. Für den STM32F4 stehen spezielle Cortex M4 Portierungen zur Verfügung. Nach der Einbindung sollten die Verzeichnisse wie in Abbildung \ref{fig:SourceTree} aussehen.
\begin{figure}[htb]
	\centering
		\includegraphics[width=0.4\textwidth]{Pictures/Einrichtung/sourceTree.png}
	\caption{Source}
	\label{fig:SourceTree}
\end{figure}


%!TEX root = FreeRtos ARM uController.tex
\subsection{Memory Allocation}
Beim Erzeugen von RTOS Objekten wie Tasks, Queues oder Semaphore wird Speicher im RAM benötigt. Für die dynamische Speicherverwaltung wird in C und C++ ge\-wöhnlich die Standard C Funktionen \verb|malloc()| und \verb|free()| verwendet. Die Funktion \verb|malloc()| dient zur Allozierung von freiem Speicher und \verb|free()| zur Freigabe von alloziertem Speicher. Für Echtzeitsysteme die auf einem RTOS aufsetzen, sind diese Funktionen aufgrund der folgende Eigenschaften\cite{MasteringFreeRTOS} ungeeignet\footnote{Heap3 stellt hier eine Ausnahme dar}:
\begin{itemize}
	\item nicht thread safe
	\item nicht deterministisch
	\item tendieren zur Fragmentierung des RAM
	\item schwer zu debuggen
	\item Bibliotheksfunktionen benötigen viel Speicher
\end{itemize}
Des Weiteren sind für einige Einsatzgebiete von embedded Anwendungen Zertifikate erforderlich. Speziell in sicherheitskritischen Anwendungen (medical, military) ist die dynamische Speicherverwaltung als eine potentielle Fehlerquelle auszuschließen. Für einen solchen Fall bietet FreeRTOS ab Version 9.0 die Möglichkeit der statischen Speicherallozierung, diese werden wir am Ende dieses Abschnitts betrachten. In FreeRTOS werden  \verb|malloc()| und \verb|free()| durch die Funktionen  
\begin{lstlisting}[label=lst:vPortMallocFree, numbers = none]
void *pvPortMalloc( size_t xSize );
\end{lstlisting}
und
\begin{lstlisting}[label=lst:vPortMallocFree, numbers = none]
void vPortFree( void *pv );
\end{lstlisting}
ersetzt. Dies hat den Vorteil, dass die Implementierung dieser Funktionen an die jeweilige Anwendung angepasst werden kann. Grundsätzlich bietet FreeRTOS fünf unterschiedliche Beispiel Implementierungen der Speicherverwaltung(Heap\_1.c bis Heap\_5.c), siehe Abbildung \ref{fig:HeapsEclipse}. 
\begin{figure}[htb!]
	\centering
		\includegraphics[width=0.2\textwidth]{Pictures/Eclipse/Heaps.png}
	\caption{Einbindung von Heap1. Heap2 bis Heap5 sind vom Build ausgeschlossen}
	\label{fig:HeapsEclipse}
\end{figure}

\begin{lstlisting}[caption={Implementierung von malloc() und free() in Heap1}, linewidth=8cm,captionpos=b, label=lst:heap1, float=hbt]
//**MALLOC**
void *pvPortMalloc( size_t xWantedSize )
{
void *pvReturn = NULL;
static uint8_t *pucAlignedHeap = NULL;
	/* Ensure that blocks are always aligned to the required number of bytes. */
	#if( portBYTE_ALIGNMENT != 1 ){
		if( xWantedSize & portBYTE_ALIGNMENT_MASK )	{
			/* Byte alignment required. */
			xWantedSize += ( portBYTE_ALIGNMENT - ( xWantedSize & portBYTE_ALIGNMENT_MASK ) );
		}
	}
	#endif
	vTaskSuspendAll();
	if( pucAlignedHeap == NULL ){
		/* Ensure the heap starts on a correctly aligned boundary. */
		pucAlignedHeap = ( uint8_t * ) ( ( ( portPOINTER_SIZE_TYPE ) &ucHeap[ portBYTE_ALIGNMENT ] ) & ( ~( ( portPOINTER_SIZE_TYPE ) portBYTE_ALIGNMENT_MASK ) ) );
	}
	/* Check there is enough room left for the allocation. */
	if( ( ( xNextFreeByte + xWantedSize ) < configADJUSTED_HEAP_SIZE ) &&
		( ( xNextFreeByte + xWantedSize ) > xNextFreeByte )	)	{
		/* Return the next free byte then increment the index past this
		block. */
		pvReturn = pucAlignedHeap + xNextFreeByte;
		xNextFreeByte += xWantedSize;
	}
	xTaskResumeAll();
	return pvReturn;
}

//**FREE**
void vPortFree( void *pv )
{
	/* Memory cannot be freed using this scheme. */
	( void ) pv;
	/* Force an assert as it is invalid to call this function. */
	configASSERT( pv == NULL );
}

\end{lstlisting}

   
Diese stellen prinzipiell schon die ge\-läu\-figsten Implementierungen zur Speicherverwaltung. Es bleibt aber auch weiterhin die Möglichkeit seine eigene Speicherverwaltung zu implementieren. In dieser Arbeit werden wir Heap1 etwas genauer betrachten um ein grund\-sätz\-liches Verständnis für die FreeRTOS Speicherverwaltung zu bekommen. Heap2 - Heap 5 werden nur kurz beschrieben und können im Detail in \cite{MasteringFreeRTOS}\cite{FreeRTOSAdvanced} nachgelesen werden.      
Wie schon am Anfang dieses Abschnitts beschrieben, werden für alle RTOS Objekte Speicher benötigt, der Speicher für Objekte wie Semaphore und Tasks wird automatisch in den Erzeugerfunktionen alloziert, in dem intern die Funktion \textit{pvPortMalloc()} aufgerufen wird. Die Erzeugerfunktion xTaskCreate() beispielsweise, erzeugt eine FreeRTOS Task. Listing \ref{lst:xTaskCreate} zeigt wie \verb|xTaskCreate()| die Funktion \verb|pvPortMalloc()| verwendet (Zeile 5, 11) um Speicher für den Stack und den Task Control Block zu allozieren.
\begin{lstlisting}[caption={xTaskCreate() memory allocation. Aus Task.c}, linewidth=8cm,captionpos=b, label=lst:xTaskCreate, float=hbt]
StackType_t *pxStack;
/* Allocate space for the stack 
used by the task being created. */
pxStack = 
( StackType_t * ) pvPortMalloc(( ( ( size_t ) usStackDepth ) 
* sizeof( StackType_t ) ) );

if( pxStack != NULL )
{
	/* Allocate space for the TCB. */
	pxNewTCB = ( TCB_t * ) pvPortMalloc( sizeof( TCB_t ) );

	if( pxNewTCB != NULL )
	{
		/* Store the stack location in the TCB. */
		pxNewTCB->pxStack = pxStack;
	}
//...
}
\end{lstlisting}
Alle Objekte die mittels pvPortMalloc() alloziert werden, darunter auch der Kernel selbst, teilen sich einen gemeinsamen Adressraum, siehe Abbildung \ref{fig:AddressSpace}. Eine Speicherzugriffsverletzung ist somit durchaus möglich. In Abschnitt \ref{sec:Memory Protection} wird gezeigt welche Möglichkeit der STM32F4 und FreeRTOS bieten um Speicherzugriffe sicherer zu gestalten.    
\begin{figure}[htb!]
	\centering
		\includegraphics[width=0.2\textwidth]{Pictures/EmbeddedCom/addressSpace.jpg}
	\caption{Adressraum FreeRTOS und Tasks}
	\quelle{Colin Walls - Embedded.com}
	\label{fig:AddressSpace}
\end{figure}    
\subsubsection{FreeRTOS Heap Implementierungen}
Bevor Objekte erzeugt werden können, muss ein Pool an Speicher für die Objekte definiert werden. Die einfachste Form einen Memory Pool zu erzeugen ist ein Array. In FreeRTOS nennt sich dieses Array ucHeap. 
\begin{lstlisting}[label=lst:ucHeap, float=hbt, numbers = none]
static uint8_t ucHeap[ configTOTAL_HEAP_SIZE ];
\end{lstlisting}
\newline
Die Größe des Heaps wird durch das Präprozessor-Define \verb|configTOTAL_HEAP_SIZE | konfiguriert. Die Gesamtgröße berechnet sich wie folgt:
\newline
\newline
MaxHeapSize $=$ configTOTAL\_HEAP\_SIZE $\ast$ Wortbreite\footnote{Beim STM32F4 ist die Wortbreite 32 bit} 
\newline
\newline
Die Speicherverwaltung durch Heap1 ist sehr einfach. Heap1 deklariert lediglich die Funktion pvPortMalloc(). Die Funktion pvPortFree() wird nicht ausimplementiert. Abbildung \ref{fig:Heap1} zeigt wie sich der Speicher nach dem Erzeugen von zwei Tasks aussieht. Für jede Task wird ein TCB und ein Stack erzeugt, die Speicherobjekte liegen direkt hintereinander, da pvPortFree() nicht implementiert ist, kommt es auch nicht zu einer Fragmentierung des Speichers. Diese lineare Speicherzuweisung gilt für alle Objekte die mittels pvPortMalloc() alloziert werden, dazu gehören sowohl RTOS spezifische Objekte als auch Objekte die durch den Benutzer erzeugt werden. 
\begin{figure}[ht!]
	\centering
		\includegraphics[width=0.3\textwidth]{Pictures/FreeRTOSOrg/heap1Alg.png}
	\caption{Beispiel Speicherbelegung nach drei Instanziierung von Tasks}
	\quelle{FreeRTOS.org}
	\label{fig:Heap1}
\end{figure}
Ein so einfacher Speicheralgorithmus wie Heap1 hat durchaus seine Berechtigung. Bei vielen embedded Anwendungen wird der Speicher für die benötigten Objekte vor dem Start des Schedulers erzeugt. Eine spätere Freigabe von belegten Ressourcen ist nicht nötig, da die Objekte über die gesamte Laufzeit des Programms bestehen sollen. Genau für solche Anwendungen steht Heap1 zur Verfügung. 
Nachfolgend ein Kurzüberblick über die nicht beschriebenen Beispiel Implementierungen.  
\begin{itemize}
	\item Heap2 - Ähnlich Heap1. Erlaubt allerdings Speicherfreigabe durch vPortFree(). Best Fit Algorithmus zur Speicherallozierung. 
	\item Heap3 - Verwendet C Library Malloc() und free() und deaktiviert den Scheduler zur Speicherallozierung.
	\item Heap4 - Ähnlich Heap1 und Heap2. Verwendet First Fit Algorithmus zur Speicherallozierung. Verbindet mehrere kleinere Speicherblöcke zu einem Großen. Minimiert Speicherfragmentierung.
	\item Heap5 - Gleicher Algorithmus wie Heap4 allerdings können mehrere Memory Pools erzeugt werden.
\end{itemize}
\subsubsection{Memory Protection}
\label{sec:Memory Protection}
STM32F4 spezifisch, MPU vorhanden
\begin{figure}[hb!]
	\centering
		\includegraphics[width=0.3\textwidth]{Pictures/EmbeddedCom/addressSpaceMMU}
	\caption{Adressraum FreeRTOS und Tasks mit Memory Protection}
	\quelle{Colin Walls - Embedded.com}
	\label{fig:AddressSpaceMMU}
\end{figure} 
\subsubsection{Static Memory Allocation}
Die statische Speicherverwaltung wird durch das Präprozessor-Define configSUPPORT\_STATIC\_ALLOCATION 1 in der FreeRTOS\_config aktiviert. Für die statische Objekterzeugung können die dynamischen Erzeugerfunktionen nicht mehr verwendet werden, daher stehen spezielle Erzeugerfuntkionen für die statische Speicherallozierung zur Verfügung, wie xTaskCreateStatic() statt xTaskCreate() oder xSemaphoreCreateBinaryStatic() statt xSemaphoreCreateBinary(). Der Vorteil der statischen Speicherverwaltung ist, dass der Belegte Speicher im RAM schon zur Übersetzungszeit bekannt ist und die potenzielle Fehlerquelle der dynamischen Speicherverwaltung vermieden wird. Nachteil ist, dass mehr RAM verwendet wird als bei den meisten Heap Implementierungen. Heap1 stellt eine geeignete Alternative in der dynamischen Speicherverwaltung.   

       
%!TEX root = FreeRtos ARM uController.tex
\subsection{Scheduling}
\label{Scheduling}
%Ich hab im Duden nachgeschlagen, es heißt wohl der Task. Ich versuche das im ganzen Dokument glatt zu ziehen
%Michael: Ok
Der Scheduler ist die Kernkomponente eines Echtzeitbetriebssystem Kernels, da er eine quasi parallele Aus\-füh\-rung von Tasks ermöglicht. Ein Task stellt dabei eine ei\-gen\-stän\-di\-ge lauffähige Programmeinheit dar. Gewöhnlich implementiert ein Task eine endlose Schleife, die bis zur ihrer Ausführung auf ein Timingevent oder ein Synchronisierungsevent blockiert (Listing \ref{lst:taskPseudo}). Die einmalige Aus\-füh\-rung einer Task wird "`job"'\cite{9780128015070} genannt. 
\begin{lstlisting}[caption={Pseudocode für die Implementierungsmuster einer periodischen (zeitgesteuert) Task und sporadischen (eventgesteuert) Task ~\protect\citeA{LorenK.Rhodes2017}}, linewidth=8cm,captionpos=b, label=lst:taskPseudo, float=hbt]
//Periodische Task
while(true)
{
   waitForStartOfNextCycle; //the period is p msec
   ExecProcess();  //takes c msec in the worst case
}

//Sporadische Task
while(true)
{
   waitForEvent;   //successive events won't occur in less than p msec
   ExecResponseProcess();   //takes c msec worst case
}
\end{lstlisting}
Abhängig vom aktuellen Zustand der Tasks und dem gewählten Scheduling\-algorithmus wählt der Scheduler den nächsten Task, der ausgeführt werden soll. Auf einem $\mu$\-Pro\-zesso\-r mit einem Kern (Prozessoreinheit) kann dabei immer nur ein Task zur Zeit ausgeführt werden. Die Ausführung ist also nicht wirklich parallel, sondern eine sequentielles Ausführen von Task jobs, die durch den Scheduler organisiert werden. Der Vorgang des Task-Wechsels durch den Scheduler wird Kontextwechsel oder Contextswitch genannt. Der Kontextwechsel beeinflusst nicht die Instruktionsfolge des Tasks. Zum Zeitpunkt der Unterbrechung wird durch den Scheduler eine Art Schnappschuss des Tasks erstellt. Alle Register und der Stack des Tasks werden gesichert. Nachdem der Scheduler den verdrängten Task wieder zur Aus\-füh\-rung ausgewählt hat, werden alle Register und der Stack wiederhergestellt und in die entsprechenden $\mu$\-Pro\-zes\-sor\-re\-gis\-ter geladen. Der Task wird danach ab der letzten Instruktion fortgeführt. Abbildung \ref{fig:ContextSwitch} zeigt wie ein Task wäh\-rend seiner Ausführung unterbrochen wird.
%Michael in einer Zeiteinheit habe ich wieder entfernt, da es das ganze mit Timeslicing und Tick interrupts durcheinader bringt. 
\begin{figure}[ht!]
	\centering
		\includegraphics[width=0.4\textwidth]{Pictures/FreeRTOSOrg/ExeContext.png}
	\caption{Der Kontextwechsel eines Tasks findet mitten in der Aus\-füh\-rung statt. Alle Register, die für die weitere Ausführung benötigt werden, werden durch den Scheduler gesichert. Bild-Quelle~\protect\citeA{MasteringFreeRtos} }
	\label{fig:ContextSwitch}
\end{figure}
\newline
Neben den User-Tasks, die durch den Entwickler erstellt werden, gibt es noch den Idle Task. Dieser wird automatisch beim Start des Schedulers erstellt. Der Idle Task hat immer die niedrigste Priorität (0) und wird immer dann ausgeführt, wenn kein User-Task zur Aus\-füh\-rung bereit steht. Der Idle Task ist ein Indikator für über\-schüss\-ige Prozessorzeit. Mittels der Idle-Hook Funktion kann dem Idle Task Funktionalität durch den Entwickler hinzugefügt werden. Wie der Idle Task zum Energiesparen genutzt werden kann wird in Abschnitt \ref{sec:Low Power Modes} beschrieben.
Folgende Zu\-stän\-de kann ein FreeRTOS Task im Scheduling annehmen: 
\begin{itemize}
	\item Running: Der Task wird zur Zeit vom Scheduler ausgeführt
	\item Blocked: Der Task ist nicht bereit und wartet auf ein Synchronisations- oder ein Timer Event
	\item Ready: Der Task ist bereit und wartet auf seine Aus\-füh\-rung durch den Scheduler
	\item Suspended: Der Task hat vTaskSuspend() aufgerufen und wurde vom Schedulingvorgang ausgeschlossen
\end{itemize}
 Abbildung \ref{fig:TaskStates} zeigt ein vollständiges Zustandsdiagramm eines FreeRTOS Tasks.
\begin{figure}[ht!]
	\centering
		\includegraphics[width=0.2\textwidth]{Pictures/FreeRTOSOrg/taskStates.png}
	\caption{Übersicht aller Task-Zustandstransitionen in FreeRTOS. Der Zustandswechsel findet entweder durch den Aufruf einer FreeRTOS API Funktion statt oder aber durch Events z.B. Interrupts, Timer-Events. Der Wechsel in den Zustand Running wird durch den Scheduler bestimmt und ist durch den Schedulingalgorithmus definiert.  Bild-Quelle~\protect\citeA{MasteringFreeRtos}}
	\label{fig:TaskStates}
\end{figure} 
\newline
Die Grundlage aller zur Verfügung stehenden Scheduling\-algorithmen ist das Round Robin Verfahren\cite{9783827373427}. Dabei werden alle lauf\-fäh\-igen Tasks (Ready) gleicher Priorität in einer Liste verwaltet. Jeder Task in der Liste erhält ein gewisses Zeitquantum\footnote{Round Robin definiert nicht die Länge des Zeitquantums}, welches bestimmt, wie lange einem Task der Prozessor zugeteilt wird. Nach Ablauf des Zeitquantums wird ein Kontextwechsel durchgeführt. Der näch\-ste Task in der Liste erhält Prozessorzeit und der ausgelaufene Task wird durch den Scheduler automatisch hinten an die Liste angefügt. Da jedem Task in FreeRTOS eine gewisse Priorität zugewiesen wird, ist auch für jede Priorität eine eigene Round Robin-Liste nötig. Dieses Verfahren wird auch Priority Scheduling \cite{9783827373427} genannt. Abbildung \ref{fig:PrioList1} veranschaulicht den Aufbau dieser Listen und in Listing \ref{lst:nextTask} wird gezeigt wie das Priority Scheduling im FreeRTOS Source Code umgesetzt wird. 
\begin{figure}[htb]
	\centering
		\includegraphics[width=0.3\textwidth]{Pictures/Scheduling/PrioList1.png}
	\caption{Aufbau der Prioritätenliste nach Round Robin in FreeRTOS. Alle aufgeführten Tasks sind bereit zur Ausführung. Task A wird aktuell durch den Scheduler ausgeführt. Nach dem Ablauf des Zeitquantums wird A hinter B einsortiert. Die Maximale Priorität wird durch configMaxPrio bestimmt. Der Idle Task wird automatisch durch den Kernel erzeugt und hat immer die niedrigste Priorität. }
	\label{fig:PrioList1}
\end{figure}
\begin{lstlisting}[caption={FreeRTOS Source zur Priroty Task Selection aus Task.c. Alle lauffähigen Tasks werden in einem Array verwaltet pxReadyTaskLists. Die Listen verwalten sich durch Referenz-Pointer in den TCBs der einzelnen Tasks}, linewidth=8cm,captionpos=b, label=lst:nextTask, float=hbt]
#define taskSELECT_HIGHEST_PRIORITY_TASK(){																									
	UBaseType_t uxTopPriority = uxTopReadyPriority;														
		/* Find the highest priority queue that contains ready tasks. */								
		while(listLIST_IS_EMPTY(&(pxReadyTasksLists[ uxTopPriority ]))){																								
			configASSERT( uxTopPriority );																
			--uxTopPriority;																			
		}																								
		/* listGET_OWNER_OF_NEXT_ENTRY indexes through the list, so the tasks of						
		the	same priority get an equal share of the processor time. */									
		listGET_OWNER_OF_NEXT_ENTRY(pxCurrentTCB, &(pxReadyTasksLists[uxTopPriority]));			
		uxTopReadyPriority = uxTopPriority;																
	} /* taskSELECT_HIGHEST_PRIORITY_TASK */
\end{lstlisting}
Dem Entwickler stehen zwei Konfigurationsmöglichkeiten des FreeRTOS Scheduler zur Auswahl: Der Scheduler kann entweder im Cooperative Mode oder im Preemptive Mode ausgeführt werden. Welchen Modus der Scheduler als Schedulingalgorithmus verwendet, wird durch das folgende Define in der FreeRTOS config bestimmt.
\begin{lstlisting}[numbers = none]
#define configUSE_PREEMPTION
\end{lstlisting}
Im Preemptive Mode wird ein aktiver Task mit niedriger Priorität sofort von einem Task mit höherer Priorität verdrängt. Ein Kontextwechsel wird durchgeführt. Im Cooperative Mode hingegen wird ein Kontextwechsel erst durchgeführt, wenn ein Task den Prozessor explizit abgibt. Die geschieht beispielsweise durch die Funktion xTaskYield() oder durch einen blockierenden API Aufruf. Abbildung \ref{fig:PreVSCo} zeigt den Vergleich beider Modi durch einen beispielhaften Ablauf. 
\begin{figure}[htb]
	\centering
		\includegraphics[width=0.4\textwidth]{Pictures/EMCUIT/PreemptiveCooperative.png}
	\caption{Im Cooperative Mode wird der Prozessor von einer Task erst abgegeben, wenn diese explizit taskYield() aufruft. Selbst wenn ein Task mit höherer Priorität in den Ready Zustand wechselt, läuft der Task mit niedrigerer Priorität weiter. Im Gegensatz dazu steht das Pre-Emptive Scheduling (hier mit Time-Slicing). Es unterbricht den laufenden Task mit niedrigerer Priorität sofort, sobald ein Task mit höherer Priorität in den Zustand Ready wechselt. Bild-Quelle~\protect\citeA{MasteringFreeRtos}}
	\label{fig:PreVSCo}
\end{figure}
Für den Preemptive Mode bietet FreeRTOS eine weitere Konfigurationsmöglichkeit. Mit der nachfolgenden Pre-Prozessordirektive lässt sich das Zeitschlitzverfahren (time slicing) aktivieren. 
\begin{lstlisting}[numbers = none]
#define configUSE_TIME_SLICING 1
\end{lstlisting}
Durch das Zeitschlitzverfahren wird die zugeteilte Prozessorzeit für Tasks gleicher Priorität gleichmäßig aufgeteilt. Dies geschieht durch Ein\-füh\-rung eines festen Tick-Interrupt Intervalls. Bei jedem Tick Interrupt wird der FreeRTOS SysTickHandler aufgerufen. Listing \ref{lst:SysTickS} zeigt die Implementierung des FreeRTOS SysTicks. Der SysTickHandler ist Bestandteil des Schedulers. Er überprüft bei jeder Ausführung, ob sich ein Task gleicher Priorität im Ready Zustand befindet. Sollte es einen solchen Task geben, wird ein Kontextwechsel durchgeführt und der Task erhält den Prozessor zugeteilt. Des Weiteren kümmert sich der SysTickHandler um die Verwaltung des TickCount, welcher als Referenz für alle RTOS Timingfunktionen dient. Abbildung \ref{fig:SysTick} zeigt diesen Vorgang nochmal im zeitlichen Verlauf.
\begin{lstlisting}[caption={FreeRTOS Source des SysTickHandlers aus Task.c. Der SysTickHandler verwaltet den TickCount. Der TickCount dient allen Timingfunktionen des RTOS Kernels als Zeitreferenz. Des Weiteren wird beim aktiven Time Slicing überprüft ob ein Kontextwechsel nötig ist. Der Kontextwechsel wird dann ggf. durch den PendSVHandler durchgeführt.}, linewidth=8cm,captionpos=b, label=lst:SysTickS, float=hbt]
void xPortSysTickHandler( void ){
	portDISABLE_INTERRUPTS();
	{
		/* Increment the RTOS tick. */
		if( xTaskIncrementTick() != pdFALSE )
		{
			/* A context switch is required. */
			portNVIC_INT_CTRL_REG = portNVIC_PENDSVSET_BIT;
		}
	}
	portENABLE_INTERRUPTS();
}
\end{lstlisting}
\begin{figure}[htb]
	\centering
		\includegraphics[width=0.4\textwidth]{Pictures/FreeRTOSOrg/TickISR.png}
	\caption{Beispielhafter Ablauf eines SysTickInterrupts.(1) kein User Task ist ready, der Idle Task ist aktiv. (2) SysTickInterrupt. (3) SysTickHandler wird aufgerufen. (4) vControlTask ist ready und ein Kontextwechsel wird durchgeführt. vControlTask hat hier die gleiche Priorität wie der IdleTask. (5)vControlTask wird ausgeführt. Bild-Quelle~\protect\citeA{MasteringFreeRtos}}
	\label{fig:SysTick}
\end{figure}
Die wohl am häufigsten verwendete Konfiguration ist der Preemptive Mode mit aktivem Zeitschlitzverfahren.
\begin{lstlisting}[numbers = none]
#define configUSE_PREEMPTION 1
#define configUSE_TIME_SLICING 1
\end{lstlisting}
Diese Einstellung wird üblicherweise Prioritized Pre-emptive Scheduling with Time Slicing genannt. Abbildung \ref{fig:timeslice} zeigt wie sich diese Konfiguration des Schedulers bei mehreren Tasks mit unterschiedlicher Priorität verhält.
\begin{figure}[htb]
	\centering
		\includegraphics[width=0.5\textwidth]{Pictures/Scheduling/timeslice2.png}
	\caption{Durch das Zeitschlitzverfahren wechseln sich Task1 und Idle Task bei jedem SysTick Interrupt ab, da beide die gleiche Priorität haben. Bei T6 ist Task 1 bereit und verdrängt (preempt) aufgrund seiner höheren Priorität Task2. Nachdem Task 1 blockiert, wird Task 2 fortgeführt. Bild-Quelle~\protect\citeA{MasteringFreeRtos}}
	\label{fig:timeslice}
\end{figure}
\subsection{Echtzeitfähigkeit}
Ein Echtzeitbetriebssystem zeichnet sich dadurch aus, dass es auf ein eintreffendes Ereignis eine deterministische und somit vorhersehbare Reaktion gibt. Dies wird bei FreeRTOS, wie bei allen anderen Echtzeitbetriebssystemen unter anderem durch geeignete Scheduling Algorithmen und nicht blockierende Interprozesskommunikation erreicht\cite{9780128015070} \cite{Jones:1997:CRT:269005.266689} \cite{Regehr:2001:ACR:882481.883779}. Prioritätengesteurtes preemtives Scheduling, wie es bei FreeRTOS verwendet wird, ist eine der Echtzeit Scheduling Strategien\cite{9780128015070}, die eine Implementierung von echtzeitfähigen Anwendungen ermöglicht. Tasks mit Echtzeitanforderungen erhalten gewöhnlich eine höhere Priorität als Tasks ohne Echtzeitanforderungen, so kann garantiert werden, dass eine Echtzeittask nicht von eine Task ohne Echtzeitanforderungen blockiert wird. Durch die Zuweisung einer höheren Priorität ist aber noch nicht sichergestellt, dass der Echtzeittask die geforderten Deadlines einhält. 
%Work in progress, bin hier noch nicht fertig. Die formulierungen sind noch unsauber.
Sind mehrere Echtzeittasks in einer Anwendung vorhanden, kann eine unpassende Prioritätenzuweisung  dazu führen, dass die Anwendung ihre Deadlines nicht einhalten kann\cite{9780128015070}. Um die passenden Prioritäten zu bestimmen, stehen mehrere Verfahren zur Verfügung, wie rate monotonic assignment (RMA) oder deadline monotonic assignment (DMA). Ob eine Anwendung "`schedulbar"' ist und ob alle geforderten Deadlines und Perioden eingehalten werden, wird gewöhnlich zu beginn in der Schedulinganalyze bestimmt. Diese umfasst unter anderem die Bestimmung worst case runtimes und die worst case response times der Tasks. Eine sehr gute Beschreibung zur Scheduling Analyse und den Verfahren zur Prioritätenvergabe finden sich in \cite{9780128015070}. FreeRTOS ist grundsätzlich für weiche Echtzeitanforderungen geeignet, dadurch dass die Auswahl der nächsten Task durch die gegeben Scheduling Algorithmen garantiert wird und somit vorhersehbar ist. Bei harten Echtzeitanforderungen kann man das nicht so pauschal sagen, da es hier auf weitere Faktoren ankommt. Wichtig ist hier beispielsweise die Periode des zugrundeliegendem SysTick Interrupt, der für alle Timings des RTOS Kernels verantwortlich ist. Dieser ist wieder Abhängig von der eingesetzten Hardware. Ein weiterer wichtiger Punkt ist, dass für FreeRTOS keine worst cast Timing Informationen für den Scheduler und API Funkionen publiziert werden. Bei weichen Echtzeitsystemen können diese Timings oft vernachlässigt werden, bei harten Echtzeitsystemen müssen diese allerdings bekannt sein, damit diese mit in die Schedulinganalyse mit einbezogen werden können. Um harten Echtzeitanforderungen in FreeRTOS gerecht zu werden, müssten diverse Messungen vorgenommen werden, um die Echtzeit-Performance zu bestimmen, dies wird beispielsweise in\cite{RealTimePerformance} gezeigt.     
%!TEX root = FreeRtos ARM uController.tex
\subsection{Intertask Kommunikation}
%Michael: Ich habe in diesen Abschnitten nichts verändert, nur ein paar Kommentare. Vielleicht sollten wir noch etwas an Beispielen einbringen, wie von Frau Ma empfohlen hat. Zur Zeit ist es ne ziemliche wall of text :) 
In Projekten, in denen verschiedene Tasks parallel verarbeitet werden, ist es häufig erforderlich, dass diese Tasks die Möglichkeit besitzen Informationen untereinander auszutauschen. Zum Einen kann ein Task Informationen produzieren, die ein anderer Task für die weitere Verarbeitung benötigt. Zum Anderen können beide Tasks gemeinsame Ressourcen (z.B. Hardwareregister oder Variablen) nutzen. Hierbei muss sichergestellt werden, dass die dort hinterlegten Daten jederzeit korrekt sind. FreeRTOS bietet hierfür verschiedene Funktionen zur Interprozesskommunikation an.\newline Zuerst sind hier die Queues zu nennen. Diese dienen dem klassischen Austausch von Informationen, indem Daten durch einen Task in die Queue hineingeschrieben werden und von einem zweiten Task gelesen werden. Die Größe der Queue wird beim Aufruf der Erzeugerfunktion definiert. In FreeRTOS wird eine Queue mit folgender Funktion erstellt: 
\begin{lstlisting}[numbers = none]
xQueueCreate(uxQueueLength,uxItemSize)
\end{lstlisting}
Es werden keine expliziten Queues angeboten, die Pointer speichern. Es ist jedoch möglich Pointer als Datum in einer Queue zu hinterlegen. Um einen Überlauf der Queue zu verhindern, werden Tasks, die in eine volle Queue hineinschreiben wollen, in den Zustand Blocked versetzt. Ebenso werden Tasks behandelt, die versuchen Daten aus einer leeren Queue abzurufen. Um die Echtzeitfähigkeit der Tasks weiter zu gewährleisten, bieten die Funktionen xQueueSendToFront(), xQueueSendToBack() und xQueueReceive() einen Parameter xTicksToWait an. Mit xTicksToWait übergibt ein Task den Zeitwert, in dem der Task eine Antwort erwartet. Erfolgt innerhalb dieser Zeit keine Antwort, so wird der Task mittels Rückgabewert über diesen Timeout informiert.
Wenn mehrere Tasks auf eine gemeinsame Queue zugreifen, wird deren Zugriff im o.g. Fall zuerst nach Priorisierung des Tasks und danach durch Wartezeit gesteuert. Je nach Zustand der Queue erhält der Task mit der höchsten Priorität den Zugriff. Existieren zwei Tasks mit der gleichen Priorität, so darf der Task zugreifen, der schon länger auf einen Zugriff wartet.\newline
Neben Queues werden von FreeRTOS Mailboxen angeboten. Diese verhalten sich grund\-sätz\-lich wie Queues, jedoch beinhalten sie nur ein Datenobjekt. Dieses wird nach dem Lesen nicht direkt gelöscht, sondern verbleibt in der Mailbox, bis es von einem datenerzeugenden Task über\-schrie\-ben wird. Mailboxen sind vor allem in Szenarien interessant, in denen mehrere Tasks lesend auf ein erzeugtes Datum zugreifen sollen. Beispielsweise greift ein Task zur Verarbeitung und ein niedriger priorisierter Task zur Anzeige auf die Mailbox zu.
%Ich konnte gar nichts zu Mailboxes finden in FreeRTOS, wo hast du das gefunden. Das einzige was ich finden konnte, war wie man Mailboxes durch Notify implementiern kann.http://www.freertos.org/RTOS_Task_Notification_As_Mailbox.html
\newline
FreeRTOS bietet außerdem Semaphore an. Üblicherweise werden diese für die Behandlung von Interrupts verwendet. Semaphore werden im Rahmen der Interprozesskommunikation meist zur Synchronisierung der Tasks angewendet. 
%Michael :Bei Semaphoren geht es doch eigentlich nicht nur um Interrupts, sondern um die Synchronisierung von Zugriffen auf gemeinsame Ressourcen? Dabei ist doch eigentlich egal ob das ganze aus einer ISR oder aus einer Task stattfindet. Der einzige unterschied ist, dass man eine andere Funktion aufrufen muss _fromISR(). Könnte mir vorstellen, dass sowas von den Profs gefragt wird.
Hierzu wird durch einen Task der Semaphor angefordert und durch den zweiten Task der Semaphor vergleichbar einem Interrupt gesetzt. Semaphore werden detailliert im Abschnitt \ref{sec:Interrupt} beschrieben.
%Semaphor vergleichbar einem Interrupt gesetzt. 
%Michael: Wie meinst du das?  Ein Interrupt wird doch durch die Hardware gesetzt, das ist doch ganz was anderes.
\newline
Die nächste Gruppe der Funktionen zur Interprozesskommunikation sind die Mutexe. Diese bilden eine Sonderform der Semaphore. 
%Hier noch ein Auszug der FreeRTOS Website, vielleicht auch noch hilfreich.
%"--Binary semaphores and mutexes are very similar but have some subtle differences: Mutexes include a priority inheritance mechanism, binary semaphores do not. This makes binary semaphores the better choice for implementing synchronisation (between tasks or between tasks and an interrupt), and mutexes the better choice for implementing simple mutual exclusion.--see http://www.freertos.org/Embedded-RTOS-Binary-Semaphores.html.
Mutexe werden benutzt um Zugriffe auf gemeinsam genutzte Ressourcen zu steuern. Wenn ein Task auf eine Ressource zugreifen will, so prüft er vorher, ob er den Mutex erhalten kann. Ist dies nicht der Fall, weil ein anderer Task den Mutex besitzt, so muss der Task warten bis der andere Task den Mutex zurück gibt. Zur Unterstützung von rekursiven Funktionen bietet FreeRTOS rekursive Mutexe an, die von einem Task mehrfach angefordert werden können. 
%Grafik Deadlock
%Ist das wirklich interessant ? Glaube das rekursive Mutexe 99,9 % der Fälle nicht eingesetzt werden. 
Im Rahmen von Echtzeitsystemen müssen jedoch zwei Risiken beim Einsatz von Mutexen berücksichtigt werden. Es besteht ein Risiko darin, dass Deadlocks entstehen. Hierbei versuchen zwei, oder mehr Tasks zeitgleich auf zwei, oder mehr Ressourcen zuzugreifen. Beiden Tasks gelingt es mindestens einen Mutex zu erhalten. In Folge kann keiner der Tasks vollen Zugriff auf die Ressourcen erlangen und wartet jeweils auf den anderen. Der Deadlock kann nur aufgebrochen werden, indem einer der Tasks seine Mutexe zurück gibt und der andere diese erhalten kann. 
Durch die notwendigen Timeouts kann die Echt\-zeit\-fähig\-keit des Systems sichergestellt werden. Das andere Risiko besteht darin, dass ein niedrig priorisierter Task einen Mutex erhält und damit einem höher priorisierten Task von der Verwendung der Ressource ausschließt. FreeRTOS bietet hierzu eine Möglichkeit den niedrig priorisierten Task kurzzeitig auf die Priorität des hoch priorisierten Tasks zu setzen, sodass eine zeitnahe Abarbeitung stattfinden kann. Es kann jedoch auch hier zu Laufzeitproblemen kommen. 
%Michael: Meinst du wir sollten erklären wie ein Deadlock und Starvation funktioniert und wie man diese auflösen kann ? Reicht es nicht wenn wir darauf hinweisen ? Das ist ja eigentlich ein allgemeines Problem und nicht RTOS spezifisch. Stattdessen könnte man doch lieber ein paar Beispiele zu den Funktionalitäten bringen.  
Die Entwickler von FreeRTOS verwenden daher Wrapper-Funktionen, auch Gatekeeper genannt. Diese nehmen eine Kapselung der Ressourcen vor und werden über Queues angesprochen. Auf diesem Weg kann auf den Einsatz von Mutexen weitestgehend verzichtet werden.
%Grafik für Wrapper-Funktion
%Michael: Habe noch nie von Gatekeepern (außer bei Thor :P ) gehört und wie helfen sie, dass wir nicht ganz klar.
Die dritte Form der Intertaskkommunikation, die von FreeRTOS angeboten wird, ist die Task Notification. Sie ist die Variante mit dem geringsten Ressourcenaufwand, da anders als bei Queues und Mutexes keine Datenobjekte angelegt werden müssen. Durch das Aktivieren der Task Notifikation innerhalb der FreeRTOSConfig.h (Setzen von configUSE\_TASK\_NOTIFICATIONS auf 1) wird pro Task ein fester Speicherbereich reserviert, der für die Notification genutzt wird. Task Notifications werden direkt an den Zieltask gesendet. Es findet, anders als bei Queues, kein Zwischenspeichern der Information statt. Wenn ein Task einen anderen Task benachrichtigt, wird er in den Zustand Blockiert versetzt, bis der Notification Wert in den hierfür vorgesehen Speicherbereich geschrieben wurde. Darüber hinaus ist bei Notifications sichergestellt, dass die Informationen ausschließlich zwischen den beiden beteiligten Tasks ausgetauscht werden. Ein Zugriff eines dritten Tasks oder einer ISR (Interrupt Service Routine) auf diese Kommunikation ist ausgeschlossen.

%!TEX root = FreeRtos ARM uController.tex
\subsection{Interrupt Handling}
\label{sec:Interrupt}
Interrupts können innerhalb von FreeRTOS auf verschiedenen Wegen behandelt werden. Hierbei bilden durch die Hardware gesteuerte Interrupt Service Routinen (ISR) die Basis. Um die Verarbeitungszeit für einen Interrupt kurz zu halten, führen ISRs gewöhnlich nur wenige Instruktionen aus. Dies geschieht beispielsweise durch das Informieren einer FreeRTOS Task mittels Intertaskkommunikation. Die FreeRTOS Task führt danach die eigentliche Aufgabe aus. Da die normalen API für den Aufruf aus einer Task implementiert wurden und spezielle Eigenschaften einer Task verwenden, kann eine normale API Funktion nicht in einer ISR verwendet werden. Beispielsweise setzen viele Intertask API Funktionen, die Task in den Blocked Zustand. Dies ist im ISR Kontext natürlich nicht möglich.
Damit man diese Funktionen dennoch nutzen kann, stellt FreeRTOS für die meisten Zugriffe auf die API, spezielle ISR API Funktionen zur Verfügung. Diese Funktionen haben den postfix FromISR. ISR API Funktionen deaktivieren kurzfristig die Interruptverarbeitung innerhalb der kritischen Zugriffe.
Damit Tasks die Mög\-lich\-keit haben auf Interrupts zuzugreifen bietet FreeRTOS verschiedene Mög\-lich\-keit\-en an. Zuerst die binären Semaphoren, die mit xSemaphoreCreateBinary(void) erstellt werden. Hierbei handelt es sich um Speichervariablen, die einen binären Wert annehmen können. In dem Moment, in dem die Variable den Wert TRUE annimmt, ändert die Task ihren Zustand von Blocked auf Ready. Die Task kann in den Wartezustand gebracht werden, indem xSemaphoreTake() aufgerufen wird. Die Semaphore selbst wird durch eine ISR gesetzt. Binäre Semaphoren werden meist zu Synchronisationszwecken zwischen einem Task und einem Interrupt eingesetzt.
Da nicht sichergestellt ist, dass die Task innerhalb der Zeitspanne, in der ein weiterer Interrupt auftreten kann, den vorhandenen Interrupt verarbeiteten kann, ist es möglich, dass ein Interrupt bei binären Semaphoren "'verloren'" geht. Abhilfe schaffen hier die Counting Semaphoren. Diese werden durch das Setzen der folgenden Pre-Prozessor Direktive in der FreeRTOS Config aktiviert.
\begin{lstlisting}[numbers = none]
#define configUSE_COUNTING_SEMAPHORES  1
\end{lstlisting}
Im Anschluss kann die Semaphore mittels
\begin{lstlisting}[numbers = none]
xSemaphoreCreateCounting(uxMaxCount,uxInitialCount) 
\end{lstlisting}
angelegt werden. Die Counting Semaphore werden hierbei als Queue angelegt, die jedoch eher wie ein Zähler funktioniert. Der Parameter uxMaxCount legt hierbei fest, ab welchem Wert ein Überlauf des Zählers erfolgt, uxInitialCount legt den Wert des Zählers nach der Initialisierung fest. Im Anschluss können die Counting Semaphoren wie binäre Semaphoren verwendet werden. Der Aufruf von xSemaphoreTake() holt hierbei ein Semaphorenobjekt aus der Queue und versetzt den Task erst in den Blocked Zustand, wenn die Queue leer ist.
Eine Möglichkeit um ganze Gruppen von Interrupts zusammenzufassen, sind die Event Groups. Hierbei geschieht die Tasksteuerung über ganze Bitmasken. Innerhalb einer Task kann eine "unblock condition" definiert werden, die beschreibt, ob der Task nur bei einer vollständig identischen Maske in den Zustand Ready wechselt, oder ob es bereits ausreicht, dass ein einzelnes Bit in der Maske gesetzt wird. Die Bedeutung der einzelnen Bits kann durch die Entwickler frei festgelegt werden. Eine Eventgroup wird mit dem Kommando xEventGroupCreate(void) erzeugt. Mittels xEventGroupSetBits(EventGroup,uxBitsToSet) werden die Bits uxBitsToSet innerhalb der Eventgroup <EventGroup> gesetzt. Diese Funktion kann auch innerhalb von Tasks aufgerufen werden, beispielsweise zum Zwecke der Tasksynchronisation. Eine Task kann sich in den Blocked Zustand versetzen, indem xEventDroupWaitForBits() aufgerufen wird. Diese Funktion erhält außerdem als Parameter die Eventgroup sowie die Bits, die beobachtet werden. Darüber hinaus wird mittels weiterer Parameter festgelegt, ob die aktuell gesetzten Bits zurückgesetzt werden sollen, ob alle Bits für die Reaktivierung des Tasks gesetzt sein müssen, und wie viele Taktzyklen der Task auf die Reaktivierung wartet.

%!TEX root = FreeRtos ARM uController.tex
\subsection{Low Power Modes auf dem STM32F4}
\label{sec:Low Power Modes}
Echtzeitbetriebssysteme werden immer häufiger in akkubetriebenen embedded Systemen eingesetzt. Solche Systeme verlangen eine effiziente Nutzung der Energieressourcen, um einen möglichst langen Betrieb zu ge\-währ\-leis\-ten. Bezogen auf den $\mu$Controller gibt es zwei Wege zur Energieeinsparung:
\begin{itemize}
	\item Heruntertakten des $\mu$Controller.
	\item Das System schlafen legen, wenn keine weiteren Aufgaben anstehen.
\end{itemize}
Das Heruntertakten des $\mu$Controllers ist unabhängig vom Einsatz eines RTOS. Aus diesem Grund wird hier nur der zweite Punkt genauer betrachtet, das Schlafenlegen des $\mu$Controllers. 
\begin{figure}[htb!]
	\centering
		\includegraphics[width=0.4\textwidth]{Pictures/STM32F4/powerConsumption.png}
	\caption{Der STM32F4 bietet diverse LowPower Modes. Die Modes haben starke Auswirkung auf die Funktionalität des $\mu$Controllers während der Schlafphase. Beispielsweise kann im Stop Mode keine UART Schnittstelle benutzt werden. Abhängig von der benötigten Peripherie, wählt der Entwickler einen dieser Modes. Die genutzte Taktfrequenz hat ebenfalls Einfluss auf die Stromaufnahme. Eine Anpassung der Takfrequenz zur Laufzeit ist ebenfalls möglich.}
	\label{fig:powerconsum}
\end{figure}
Abbildung \ref{fig:powerconsum} zeigt wie sich die Stromaufnahme beim STM32F4 von 40mA im Normalbetrieb(bei 168 MHz) auf 2,2$\mu$A im Tiefschlafmodus reduzieren lässt. 
In einfachen Anwendungen ist der Zustand, in dem ein Gerät schlafen gehen kann, relativ leicht zu ermitteln. In komplexen Systemen ist die Ermittlung dieses Zustands aufwendiger, da mehrere Tasks mög\-li\-cherweise auf unterschiedliche Ressourcen warten. Ein Echtzeitbetriebssystem, wie FreeRTOS, kann hierbei unterstützen. Im folgenden Abschnitt wird gezeigt, welche Funktionen FreeRTOS zur Ver\-fü\-gung stellt, um einen energieeffizienten Betrieb zu ge\-währ\-leis\-ten. Eine Mög\-lich\-keit ist die Idle-Hook Funktion. Wie bereits in Abschnitt \ref{Scheduling} beschrieben, wird der IDLE Task von FreeRTOS aktiviert, sobald sich alle User-Tasks im Blocked Zustand befinden. Durch konfigurieren des Präprozessor-Defines:        
\begin{lstlisting}[label=lst:defineIdleHook, numbers = none]
#define configUSE_IDLE_HOOK  1; 
\end{lstlisting}
kann die Idle-Hook Funktion aktiviert werden. Diese wird immer aufgerufen, sobald der Idle Task in den Zustand Running wechselt. Die Funktionalität der Idle-Hook Funktion kann frei vom Entwickler implementiert werden. 
\begin{lstlisting}[caption={Pseudocode für eine Idle-Hook Funktion.},captionpos=b, label=lst:xIdleHookExamp, float=hbt!]
extern "C" void vApplicationIdleHook( void ){
	/* Systick Interrupt deaktivieren */
	SysTick->CTRL &= ~SysTick_CTRL_TICKINT_Msk;
	//RTC konfigurieren
	setRTCWakeupTime();
	//externen Interrupt durch RTC aktivieren
	enableRTCInterrupt();
	//deaktiviere alle anderen Interrupt Quellen
	deactivateExternalDevices();
	setAllGPIOsToAnalog(); 
	disableGPIOClocks();
	//MCU stoppen und schlafen ZzZZz
	HAL_PWR_EnterSTOPMode(PWR_LOWPOWERREGULATOR_ON, PWR_STOPENTRY_WFI); 
	//Aufgewacht...the show must go on
	//aktiviere Systick
	SysTick->CTRL |= SysTick_CTRL_TICKINT_Msk;
	//reaktiviere GPIO Clocks
	enableGPIOClocks();
	//reaktiviere Externe Interrupt Quellen
	enableExternalInterrupts();	
}
\end{lstlisting}
\newline  %Sonst zerreist es das Listing über die Abbildung
Listing \ref{lst:xIdleHookExamp} zeigt Pseudocode zu einer beispielhaften Implementierung der Idle-Hook Funktion. Bevor das System schlafen gelegt werden kann, müssen alle GPIOs und IRQs konfiguriert werden, sodass das System nicht unnötigerweise aufwacht. Des Weiteren werden alle nicht benötigten GPIOs auf Analog gestellt um Energie zu sparen. Als einzige Interrupt-Quelle wird hier eine externe RTC konfiguriert. Mit dem Aufruf von HAL\_PWR\_EnterSTOPMode() wird der $\mu$Controller in den Schlafmodus versetzt. Die Funktion wird erst wieder verlassen, wenn der externe Interrupt der RTC ausgelöst wurde. Danach werden alle GPIOs auf den ursprünglichen Zustand zurückgesetzt. Außerdem sind die User-Tasks zu informieren, z.B. mittels Notify oder Message, damit das System nicht beim nächsten Tick Interrupt den Idle Task reaktiviert. Nachteil dieser Variante ist, dass die Nutzung von Software Timern nicht mehr möglich ist. Der FreeRTOS Kernel würde in diesem Fall die Idle-Hook Funktion aufrufen und sich schlafen legen, obwohl noch Software Timer aktiv sind. Die Nutzung von absoluten Zeiten ist ebenfalls nicht mehr möglich, da nach der Deaktivierung des Tick Interrupts der Tickcount nicht mehr korrekt ist. Abhilfe schafft hier eine weitere Funktionalität die FreeRTOS zur Verfügung stellt, der sogenannte Tickless Idle Mode. Der Tickless Idle Mode kann durch das folgende Define in der FreeRTOSconfig.h aktiviert werden:  
\begin{lstlisting}[label=lst:defineTicklessIdle, numbers = none]
#define configUSE_TICKLESS_IDLE  1; 
\end{lstlisting}
Im Gegensatz zur gewöhnlichen Idle-Hook Funktion be\-rück\-sich\-tigt der Tickless Idle Mode alle ausstehenden Timerfunktionen. Dazu gehören neben allen Software Timern auch Tasks, die nur für eine gewisse Zeit blockiert sind, z.B. durch die Funktion xTaskWaitUntil(). Des Weiteren muss der Tickinterrupt nicht, wie in Listing \ref{lst:xIdleHookExamp} dargestellt, explizit abgeschaltet werden. Dies wird automatisch durch den Kernel gehandelt. Der Tickless Idle Mode bietet durch die Funktion vTaskStepTick() auch die Möglichkeit den TickCount nach dem Aufwachen anzupassen. Die verpassten TickCounts können beispielsweise durch einen externen Timer oder die RTC bestimmt werden. So ist es auch möglich Software Timer für absolute Zeiten zu nutzen. Details und Beispielimplementierungen hierzu finden sich unter \cite{FreeRtosAdvanced}.  
\section{FreeRTOS in der Praxis} 
%!TEX root = FreeRtos ARM uController.tex
\subsection{Komplexität durch Nebenläufigkeit - Debugging von Echtzeitsystemen}
\label{sec:Debugging von Echtzeitsystemen}
Under Construction :D

Durch die Einsatz eines Echtzeitbetriebssystems erhält der Entwickler einige Vorteile die bereits in Abschnitt \ref{sec:Echtzeitsysteme} beschrieben wurden. Im Gegenzug entstehen aber durch die Nebenläufigkeit neue mögliche Fehlerquellen. Viele dieser Fehler, lassen sich nicht einfach analysieren und enden beispielsweise im HardFault Handler. Welche Hilfsmittel einem Entwickler speziell bei FreeRTOS (auf STM32F4) zur Verfügung stehen und welche Fehler häufig auftreten ist der Inhalt dieses Abschnitts. Die Arten von Fehler die in einem Echtzeitsystem häufig auftreten lassen sich grob in zwei Kategorien aufteilen:
\begin{itemize}
	\item Stackoverflow
	\item Synchronisatiosfehler
\end{itemize}
Bekannte Probleme detailliert erklärt in \cite{9783827373427}
\begin{itemize}
	\item dining philosopherproblem
	\item reader and writers problem
	\item producer consumer problem (starvation)
\end{itemize}
     

\begin{figure}[ht!]
	\centering
		\includegraphics[width=0.5\textwidth]{Pictures/Segger/systemview.png}
	\caption{Segger Systemview - Not referenced yet}
	\label{fig:Systemview}
\end{figure}
\begin{figure}[ht!]
	\centering
		\includegraphics[width=0.3\textwidth]{Pictures/Segger/SystemViewTarget.png}
	\caption{Segger Systemview Target - Not referenced yet}
	\label{fig:SystemviewTarget}
\end{figure}
\begin{figure}[ht!]
	\centering
		\includegraphics[width=0.5\textwidth]{Pictures/Segger/freertosThreadAwareness}
	\caption{Segger Thread Awareness- Not referenced yet}
	\label{fig:ThreadAware}
\end{figure}
%!TEX root = FreeRtos ARM uController.tex
\subsection{Echtzeitanalyse}
Uff :)
\label{sec:Echtzeitanalyse} 
%!TEX root = FreeRtos ARM uController.tex
%\pagebreak
\section{Zusammenfassung}
FreeRTOS kann als freies, professionelles Echtzeitbetriebssystem betrachtet werden. Im Vergleich zu kommerziellen Echtzeitbetriebssystemen wird ein annähernd gleicher Funktionsumfang gewährleistet. Bei den Herstellern von $\mu$\Controllern und ISP ist FreeRTOS eines der Standard Echtzeitbetriebssysteme. Es stehen gewöhnlich viele Beispiele oder Template Projekte für FreeRTOS zur Verfügung.  Besonders für Einsteiger ist FreeRTOS sehr zu empfehlen, da es kostenlos und sehr gut dokumentiert ist. FreeRTOS wird als offener Source Code zur Verfügung gestellt. Hierdurch ist es dem Entwickler auch möglich einen Blick in die Implementierung des Echtzeitsystems zu werfen. Dies ist besonders beim Verstehen des Kernels hilfreich. Da FreeRTOS auch in einer kommerziellen Version angeboten wird, kann davon ausgegangen werden, dass der Kernel auch langfristig Support erfährt. Ein Nachteil ist die komplizierte Einrichtung einer freien Entwicklungsumgebung wie Eclipse CDT. Um eine erste Beispielanwendung mit FreeRTOS auf dem Zielsystem zu implementieren, müssen diverse Konfigurationen an der Entwicklungsumgebung vorgenommen werden.

\pagebreak
\bibliographystyle{abbrv}
\bibliography{literatur} % Daten aus der Datei literatur.bib verwenden.
\end{document}

