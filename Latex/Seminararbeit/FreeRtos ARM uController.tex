% !TEX encoding = UTF-8 Unicode

% Beispiel für ein LaTeX-Dokument im Format "seminarvorlage"
\documentclass[ngerman]{seminarvorlage}
% ngerman = Deutsch in neuer Rechtschreibung, alternativ english
\usepackage[utf8]{inputenc} % Kodierung der Umlaute
\usepackage{babel} % automatische Sprachanpassung, Sprache siehe oben
\usepackage{cleveref} % für bequeme Referenzen, siehe \cref unten

%Own Code
\newcommand*{\fullref}[1]{\hyperref[{#1}]{\autoref*{#1} \nameref*{#1}}} % One single link
\newcommand*{\quelle}{% 
  \footnotesize Quelle: 
} 

\begin{document}

% Unbedingt angeben: Titel, Autoren, E-Mail
% Freiwillig: Adresse
\title{Embedded Realtime OS FreeRTOS auf STM32F4}
\numberofauthors{2}
\author{
  \alignauthor Michael Ebert\\
		\affaddr{Ad-hoc Networks GmbH}\\
    \email{ebert@ad-hoc.network}
  \alignauthor Christoph Bläßer\\
    \email{christph.blaesser@gmx.de}
}

\maketitle


\keywords{RTOS, ARM , STM32, Real Time.}

\abstract{
Im Rahmen des vorliegenden Papers wird das Echtzeitbetriebssystem FreeRtos vorgestellt. Hierzu werden zu Beginn die allgemeinen Vorgaben für Echtzeitbetriebssysteme beschrieben. Im Verlauf des Textes wird an ausgewählten Beispielen dargestellt, wie FreeRos diese Anforderungen berücksichtigt und durch geeignete Programmfunktionen umsetzt.
}
\section{Echtzeitsysteme}
Mit der steigenden Leistungsfähigkeit von modernen $\mu$ Prozessoren, steigen auch die Anforderungen an die Software die auf diese Systeme aufsetzt. Viele dieser Systeme verlangen trotz ihrer Komplexität, dass Teile des Programmablauf in bestimmten zeitlichen Grenzen ausgeführt wird und somit vorhersehbar und deterministisch sind. Systeme die eine solche Anforderung unterliegen werden Echtzeitsysteme genannt. Echtzeitsysteme unterliegen einer weiteren Unterteilung in weiche und harte Echtzeitsysteme (soft / hard realtime systems). Ein weiches Echtzeitsysteme soll eine Aufgabe in den vorgegeben zeitlichen Grenzen ausführen, ein überschreiten ist aber erlaubt und führt nicht unmittelbar zu einem Fehler. Ein hartes Echtzeitsystem hingegen muss die gestellte Aufgabe in den vorgegebenen Grenzen ausführen. Eine Überschreitung macht das System unbrauchbar. Der Programmablauf eingebetteter Systeme lässt sich auf drei Grundmodelle zurückführen, siehe Abbildung \ref{fig:Programmablauf}. Eingebettete Anwendungen können in einer einzigen Schleife (mit und ohne Interrupt Unterbrechungen) laufen oder aber in event-gesteuerten nebenläufigen eigenständigen Programmabschnitten (Thread oder Task) ausgeführt werden. Diese Nebenläufigkeit ist nur durch einen RTOS Kernel (Scheduler) möglich. Ein RTOS Kernel abstrahiert Timing Informationen und kümmert sich darum, dass die nächste Task rechtzeitig ausgeführt wird. Der Entwickler ist dafür verantwortlich, dass die Task die gewünschte Aufgabe im zeitlichen Rahmen ausführt. Wie sichergestellt werden kann, dass Task  harten bzw. weichen Echtzeitanforderungen entsprichet wird Abschnitt \ref{sec:Echtzeitanalyse} Für viele kleine Anwendungen kann die Nutzung einer einzigen Schleife durchaus sinnvoll sein, sollten beispielsweise die Ressourcen so knapp sein, dass ein Overhead an Funktionalität ausgeschlossen werden muss. Ein großer Nachteil der "`einschleifen Variante"' ist die permanente Nutzung des Prozessors. Besonders bei akkubetriebenen Geräten wie IoT Device oder Mobiltelefonen wird sehr genau auf die Energieaufnahme geachtet. Ein RTOS bietet hingegen Funktionen mit denen sehr leicht ermittelt werden kann, ob ein Gerät in einen Schlafmodus wechseln kann, dies wird in Abschnitt \ref{sec:Low Power Modes} an Beispielen von FreeRtos und einem ARM $\mu$Prozessor demonstriert. Neben der Echtzeitfähig gibt es aber noch viele weitere Vorzüge für den Einsatz eines Echtzeitbetriebssystems.  
Durch das Herunterbrechen der Anwendung in Task entstehen viele kleine Module, die jeweils eine kleine Teilaufgabe des Gesamtsystems übernehmen. Die Anwendungen werden so skalierbar und erweiterbar. Die Entwicklungsarbeit lässt sich leicht auf Teams verteilen, die unabhängig ihre Module (Tasks) entwickeln und testen. Dies ermöglicht auch den Einsatz von agilen Entwicklungsmethoden wie Scrum in der Entwicklung von eingebetteten Systemen. 
\newline
TODO Überleitung FreeRtos
\newline  
\begin{figure}
	\centering
		\includegraphics[width=0.3\textwidth]{Pictures/EmbeddedCom/cwrtos2f5c.jpg}
	\caption{Übersicht Programmabläufe}
	\quelle\url{http://www.embedded.com/}
	\label{fig:Programmablauf}
\end{figure}

\section{FreeRtos} 
\subsection{Geschichte}
\subsection{Zielsysteme}
\subsection{Entwicklungsumgebung}
\subsection{Ordner und Dateistrukturen} 
\subsection{Scheduling}
\subsection{Memory Management}
\subsection{Interprozess Kommunikation}
\subsection{Software Timer}
\subsection{Interrupt Handling}
\subsection{Low Power Modes auf STM32F4 (ARM Cortex M3)}
\label{sec:Low Power Modes} 
\section{Echtzeitanalyse}
\label{sec:Echtzeitanalyse} 
\section{Debugging von Echtzeitsystemen} 
\section{Echzeit Pattern} 
\section{Komplexität durch Nebenläufigkeit} 
\section{Zusammenfassung}
\pagebreak
\bibliographystyle{abbrv}
\bibliography{literatur} % Daten aus der Datei literatur.bib verwenden.

\end{document}

