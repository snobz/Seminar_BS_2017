% !TEX encoding = UTF-8 Unicode

% Beispiel für ein LaTeX-Dokument im Format "seminarvorlage"
\documentclass[ngerman]{seminarvorlage}
% ngerman = Deutsch in neuer Rechtschreibung, alternativ english

\usepackage[utf8]{inputenc} % Kodierung der Umlaute
\usepackage{babel} % automatische Sprachanpassung, Sprache siehe oben
\usepackage{cleveref} % für bequeme Referenzen, siehe \cref unten

\begin{document}

% Unbedingt angeben: Titel, Autoren, E-Mail
% Freiwillig: Adresse
\title{Embedded Realtime on ARM uC - FreeRTOS auf STM32F4}
\numberofauthors{2}
\author{
  \alignauthor Michael Ebert\\
    \affaddr{Rotdornweg 28}\\
		\affaddr{Ad-hoc Networks GmbH}\\
    \affaddr{25451 Quickborn}\\
    \email{ebert@ad-hoc.network}
  \alignauthor Christoph Bläßer\\
    %\affaddr{FernUniversität in Hagen}\\
		%\affaddr{Bonn}\\
    %\affaddr{58084 Hagen}\\
    \email{christph.blaesser@gmx.de}
}

\maketitle


\keywords{RTOS, ARM , STM32, Real Time.}

% Section-Überschriften werden in GROSSBUCHSTABEN umgestellt
\section{Einführung Echzeitbetriebsysteme}
\section{Vorteile eines Echzeitbetriebssystems}
\section{FreeRtos Grundlange} 
\subsection{Ordner und Dateistrukturen} 
\subsection{Scheduling}
\subsection{Interprozess Kommunikation}
\subsection{Memory Management}
\section{FreeRtos Implementierung} 
\section{Echtzeitanalyse} 
\section{Debugging von Echzeitsystemen} 


\abstract{
Im Rahmen des vorliegenden Papers wird das Echtzeitbetriebssystem FreeRTos vorgestellt. Hierzu weerden zu Beginn die allgemeinen Vorgaben für Echtzeitbetriebssysteme beschrieben. Im Verlauf des Textes wird an ausgewählten Beispielen dargestellt, wie freeRTos diese Anforderungen berücksichtigt und durch geeignete Programmfunktionen umsetzt. \cite{acmcategories,Ivory2001}.
}
\section{Übersicht und Gründe für den Einsatz eines RTOS}
Mit der steigenden Leistungsfähigkeit von modernen uProzessoren, steigen auch die Anfordernungen an die Software die auf diese Systeme aufsetzt.
Viele dieser Anwendungen verlangen trotz ihrer Komplexität, dass der Programmablauf der Software oder zumindest einige Teile dieser Software in bestimmten zeitlichen Grenzen ausgeführt wird und somit vorhersehbar und deterministisch ist.
Ein Echzeitbetriebssystem (RTOS) bietet einem Entwickler die Möglichkeit ein solches System zu entwerfen. Neben der Echzeifähigkeit gibt es aber noch viele weitere Vorzüge für den Einsatz eines Echzeitbetriebssystems. Durch die Nebenläufigkeit etstehen viele kleine Module, die als seperate Anwendungen laufen (Hier ein Beispiel mit einem Board und Komponenten). Diese Module können in Teams entwickelt werden und eigenständigen Tests unterzogen werden. Ein häufiges Problem in embedded Systemen in die Problematik des Hardwarenahen Testens. Ein weiterer sehr wichtiger Punkt ist die Entwicklung von Energieeffizienten systemen. in einer anwendung die auf einem echzeitbetriebssystem aufsetzt, lässt sich durch die ereignisorientierte entwicklung sehr schnell feststellen, ob alle Aufgaben abgearbeitet wurden, so dass man ggf. das System schlafen legen kann. Wie so etwas in FreeRtos umgesetzt wurde und wie es für eine Beispielhafte Anwendung implementiert werden kann wird in Kapitel x detailiert erläutert.  




\begin{table}[hp]\large
\begin{center}
\begin{tabular}{|c|c|}
\hline
Einhaltung von Prozessdeadline & Komplexität \\
\hline
Skalierbarkeit & Einarbeitungszeit\\
\hline
Erweiterbarkeit & Performance bei aufwändigen Berechnungen\\
\hline
Timing Abstraktionl & Aufwändige Debugging\\
\hline
Einfache Teamarbeit & \\
\hline
effiziente Energieverwaltung & \\
\hline
3erd Party driver& \\
\hline
Gummi & Bär \\
\hline
\end{tabular}
\end{center}
\caption{Vor und Nachteile eines Echzeitbetriebssystems}
\label{tttabelle}
\end{table}


\section{Zusammenfassung}
% Eine neue Spalte anfangen mit
\pagebreak

% Bibliographie entweder direkt hier eingeben (nur im Notfall)...
%
\begin{thebibliography}{9}
\bibitem{acmcategories}
%How to classify works using ACM's computing classification system.
\newblock \url{http://www.acm.org/class/how_to_use.html}.
%
\bibitem{Ivory2001}
M.~Y. Ivory and M.~A. Hearst.
\newblock The state of the art in automating usability evaluation of user
  interfaces.
\newblock {\em ACM Comput. Surv.}, 33(4):470--516, 2001.

\end{thebibliography}

% ... oder die Bibliographie mit Hilfe von BibTeX generieren,
% dies ist auf jeden Fall die bessere Lösung und sollte nach
% Möglichkeit immer verwendet werden:
\bibliographystyle{abbrv}
\bibliography{literatur} % Daten aus der Datei literatur.bib verwenden.

\end{document}

