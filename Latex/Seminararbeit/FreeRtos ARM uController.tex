% !TEX encoding = UTF-8 Unicode

% Beispiel für ein LaTeX-Dokument im Format "seminarvorlage"
\documentclass[ngerman]{seminarvorlage}
% ngerman = Deutsch in neuer Rechtschreibung, alternativ english

\usepackage[utf8]{inputenc} % Kodierung der Umlaute
\usepackage{babel} % automatische Sprachanpassung, Sprache siehe oben
\usepackage{cleveref} % für bequeme Referenzen, siehe \cref unten

\begin{document}

% Unbedingt angeben: Titel, Autoren, E-Mail
% Freiwillig: Adresse
\title{Embedded Realtime on ARM uC - FreeRTOS auf STM32F4}
\numberofauthors{2}
\author{
  \alignauthor Michael Ebert\\
    \affaddr{Rotdornweg 28}\\
		\affaddr{Ad-hoc Networks GmbH}\\
    \affaddr{25451 Quickborn}\\
    \email{ebert@ad-hoc.network}
  \alignauthor Christoph Bläßer\\
    %\affaddr{FernUniversität in Hagen}\\
		%\affaddr{Bonn}\\
    %\affaddr{58084 Hagen}\\
    \email{christph.blaesser@gmx.de}
}

\maketitle

\abstract{Die Geschichte der Gummi\-bär\-chen ist voller
Über\-raschun\-gen\ldots} % Trennhilfe \- manchmal nützlich

\keywords{RTOS, ARM , STM32, Real Time.}

% Section-Überschriften werden in GROSSBUCHSTABEN umgestellt
\section{Einleitung}

Im Rahmen des vorliegenden Papers wird das Echtzeitbetriebssystem FreeRTos vorgestellt. Hierzu weerden zu Beginn die allgemeinen Vorgaben für Echtzeitbetriebssysteme beschrieben. Im Verlauf des Textes wird an ausgewählten Beispielen dargestellt, wie freeRTos diese Anforderungen berücksichtigt und durch geeignete Programmfunktionen umsetzt. \cite{acmcategories,Ivory2001}.

\section{Und Erwachsene ebenso?}

Das ist die Frage, die interessanterweise bereits
im 19.~Jahrhundert aufgeworfen wurde~\cite[S.~237f]{Ivory2001}.

In \cref{niko} sehen wir ein Beispiel.

\begin{figure}[hp]
\begin{center}
\unitlength8mm % heir skalieren, falls gewünscht
\begin{picture}(4,6)
%außen
\put(0,0){\line(1,0){4}}
\put(4,0){\line(0,1){4}}
\put(4,4){\line(-1,1){2}}
\put(2,6){\line(-1,-1){2}}
\put(0,4){\line(0,-1){4}}
%innen
\put(0,0){\line(1,1){4}}
\put(4,4){\line(-1,0){4}}
\put(0,4){\line(1,-1){4}}
\end{picture}
\end{center}
\caption{Das Haus des Nikolaus in seiner ersten, ursprünglichen Form,
         siehe auch~\protect\cite[S.~93]{Ivory2001}.}
\label{niko}
\end{figure}

Und in \cref{tttabelle} ist das Ganze tabellarisch dargestellt.

\begin{table}[hp]\large
\begin{center}
\begin{tabular}{|c|c|c|}
\hline
Material & Tierart & essbar\\
\hline
Gummi & Bär & ja\\
\hline
\end{tabular}
\end{center}
\caption{Eine Übersicht zu den Fachbegriffen.}
\label{tttabelle}
\end{table}

\section{Zusammenfassung}
In dieser Abhandlung wurde die Geschichte neu interpretiert.
Es ergaben sich völlig neuartige Forschungsansätze, die so
vielfältig sind, dass sich die Auswirkungen gegen\-wärtig %Trennhilfe \-
kaum abschätzen lassen. Auch Artikel mit vielen Autoren~\cite{Black1988}
befassen sich mit diesem Thema.

% Eine neue Spalte anfangen mit
\pagebreak

% Bibliographie entweder direkt hier eingeben (nur im Notfall)...
%
\begin{thebibliography}{9}
\bibitem{acmcategories}
%How to classify works using ACM's computing classification system.
\newblock \url{http://www.acm.org/class/how_to_use.html}.
%
\bibitem{Ivory2001}
M.~Y. Ivory and M.~A. Hearst.
\newblock The state of the art in automating usability evaluation of user
  interfaces.
\newblock {\em ACM Comput. Surv.}, 33(4):470--516, 2001.

\end{thebibliography}

% ... oder die Bibliographie mit Hilfe von BibTeX generieren,
% dies ist auf jeden Fall die bessere Lösung und sollte nach
% Möglichkeit immer verwendet werden:
\bibliographystyle{abbrv}
\bibliography{literatur} % Daten aus der Datei literatur.bib verwenden.

\end{document}

