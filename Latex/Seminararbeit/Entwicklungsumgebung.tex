%!TEX root = FreeRtos ARM uController.tex
\subsection{Entwicklungsumgebung}
FreeRTOS ist im Prinzip nicht an eine spezielle Entwicklungsumgebung gebunden. Bevor eine Entwicklung be\-ginnt ist es dennoch ratsam sich einen Überblick über die ver\-fügbaren IDEs\footnote{Integrated Development Environment} zu machen. Der wichtigste Punkt der hier zu nennen ist, ist das Debugging. Da ein Echtzeitbetriebssystem eine weitere Abstraktionsebene hinzufügt und wie eine Art Middleware fungiert, lassen Sich viele RTOS spezifische Funktionen und Eigenschaften wie Queues, Task Stacks etc. nur mühsam mit einem Debugger wie GDB untersuchen. Viele der marktgängigen Entwicklungsumgebungen bieten daher spezielle RTOS-aware Pakete, so dass ein einfacherer Zugriff auf RTOS Objekte und Eigenschaften möglich ist. Wie die RTOS awareness beim Debugging eingesetzt wird und welche Funktionalitäten sie einem Entwickler bietet wird in Abschnitt \ref{sec:Debugging von Echtzeitsystemen} aufgezeigt. Ein weiterer Punkt der bei der Auswahl der IDE getroffen werden muss sind die Kosten. Bei Propritäre IDEs können oft mehrere tausend Euro Lizenzkosten anfallen, bieten aber den Vorteil der nahtlosen Einbindungen von $\mu$Prozessoren und Echtzeitbetriebssystem (RTOS awareness). Bei der Entwicklung von ARM uProzessoren sind hier Keil (Arm), IAR Workbench und True Studio (Atollic) zu nennen. Diese Entwicklungsumgebungen lassen sich zum Teil auch frei verwenden, allerdings mit starken Einschränkungen wie z.B. maximal Codesize. Auf der nicht proprietären Seiten steht Eclipse CDT zur Verfügung, es ist komplett frei in der Verwendung und hat keine Beschränkungen. Nachteil ist hier, dass die Integration nicht so einfach ist wie bei den proprietären IDEs. RTOS awarness wird bei Eclipse durch die Installation weiterer Plugins erreicht. Ein weiterer Nachteil ist, dass es keine Beispielprojekte für Eclipse CDT und FreeRTOS zur Ver\-fü\-gung stehen, daher müssen Projekte von Grund auf selbst konfiguriert und installiert werden. Da im Laufe dieser Arbeit Eclipse CDT für alle Beispiele verwendet wird, wird in Abschnitt \ref{sec:Einrichtung und Konfiguration} das Aufsetzen einer Basiskonfiguration erklärt. 
