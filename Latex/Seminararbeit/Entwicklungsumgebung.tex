%!TEX root = FreeRtos ARM uController.tex
\subsection{Entwicklungsumgebung}
\label{ref:Entwicklungsumgebung}
FreeRTOS ist grundsätzlich nicht an eine spezielle Entwicklungsumgebung gebunden. Dies liegt vor allem daran, dass FreeRTOS in Form von C-Quellcodedateien zur Verfügung gestellt und vergleichbar einer dynamischen Bibliothek in die zu entwickelnde Software integriert wird. Die verwendete Entwicklungsumgebung muss einen geeigneten Compiler für das Zielsystem zur Verfügung stellen. Vor dem Start eines Entwicklungsprojektes ist es dennoch ratsam sich einen Überblick über die ver\-fügbaren IDEs\footnote{Integrated Development Environment} zu machen. Der wichtigste Punkt, der hierbei zu berücksichtigen ist, ist das Debugging. Da ein Echtzeitbetriebssystem eine weitere Abstraktionsebene hinzufügt und wie eine Art Middleware fungiert, lassen sich viele RTOS-spezifische Funktionen und Eigenschaften wie Queues, Task Stacks etc. nur mühsam mit einem Debugger wie GDB untersuchen. Viele der markt\-gäng\-igen Entwicklungsumgebungen bieten daher spezielle RTOS-aware Pakete. Ein einfacherer Zugriff auf RTOS Objekte und Eigenschaften ist somit möglich. Das Debugging mittels der RTOS-awareness und die Funktionalitäten, die einem Entwickler bereitstehen, werden in Abschnitt \ref{sec:Debugging von Echtzeitsystemen} aufgezeigt. Ein weiterer Punkt, der bei der Auswahl der IDE betrachtet werden muss, sind die Kosten. Bei proprietären IDEs können oft mehrere tausend Euro Lizenzkosten anfallen. Diese bieten aber den Vorteil der nahtlosen Einbindungen von $\mu$Prozessoren und Echtzeitbetriebssystemen (RTOS\- aware\-ness). Bei der Entwicklung von ARM $\mu$\-Prozessoren sind hier Keil (ARM), IAR Workbench und True Studio (Atollic) zu nennen. Diese Entwicklungsumgebungen lassen sich zum Teil auch frei verwenden, allerdings mit starken Einschränkungen, wie z.B. der maximalen Codegröße. Ein Gegenspieler zu den proprietären IDEs ist Eclipse CDT. Es ist komplett frei in der Verwendung und hat keine Beschränkungen. Der Nachteil ist, dass die Integration nicht so einfach ist, wie bei den proprietären IDEs. RTOS-awarness wird bei Eclipse durch die Installation weiterer Plugins erreicht. Ein weiterer Nachteil sind die fehlenden Beispielprojekte für Eclipse CDT in der Kombination mit FreeRTOS. Daher müssen Projekte meist von Grund auf selbst konfiguriert und installiert werden. Da im Laufe dieser Arbeit Eclipse CDT für alle Beispiele verwendet wird, ist in Abschnitt \ref{sec:Einrichtung und Konfiguration} das Aufsetzen einer Basiskonfiguration erklärt.