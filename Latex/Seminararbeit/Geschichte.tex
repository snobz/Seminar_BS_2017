%!TEX root = FreeRtos ARM uController.tex
\subsection{Geschichte}
FreeRTOS wird seit etwa 10 Jahren von der Firma Real Time Engineers Ltd. in Zusammenarbeit mit verschiedenen Chipherstellern entwickelt. Derzeit unterstützt es 35 Architekturen und wurde mehr als 113.000 mal heruntergeladen. Das Entwicklerteam, unter Führung des Gründers Richard Barry, konzentriert sich bei der Entwicklung darauf, sowohl ein geeignetes Qualitätsmanagement umzusetzen, als auch die Verfügbarkeit der verschiedenen Dateiversionen zu gewährleisten. FreeRTOS wird in zwei verschiedenen Lizenzmodellen angeboten, die eine Anpassung der originären GNU General Public Licence darstellen. Die Open Source Lizenz (FreeRTOS) umfasst keine Garantien und keinen direkten Support. Entwickler, die diese freie Lizenz verwenden und Än\-der\-ungen am RTOS Kernel vornehmen, müssen den Quellcode ihrer Än\-der\-ungen für die Community offenlegen. In der kommerziellen Lizenz \newline (OpenRTOS) können solche Änderungen als closed source vertrieben werden. Kunden mit einer kommerziellen Lizenz bietet Real Time Engineers Unterstützung bei der Entwicklung von Projekten und Treibern. Des Weiteren werden entsprechende Garantien für die Echt\-zeit\-fähig\-keit von OpenRTOS gegeben. Real Time Engineers bietet zu FreeRTOS diverse Erweiterungen an. Ge\-führt werden diese Erweiterungen unter dem Namen FreeRTOS Ecosystem, dazu gehören unter anderem:
\begin{itemize}
	\item ein FAT Dateisystem
	\item TCP/ UDP Stacks
	\item TLS/SSL Implementierungen
\end{itemize}
Abschließend sei auch die Variante SafeRTOS erwähnt, welche durch den TÜV Süd entsprechend der Richtlinien IEC 61508 SIL 3 vorzertifiziert wurde. Die Erkenntnisse aus dieser Zertifizierung wurden in die Versionen FreeRTOS und OpenRTOS teilweise zurückgeführt.
%Quellen: http://www.freertos.org/RTOS.html (About FreeRTOS und Unterseiten) sowie http://www.freertos.org/FreeRTOS-Plus/Safety_Critical_Certified/SafeRTOS.shtml